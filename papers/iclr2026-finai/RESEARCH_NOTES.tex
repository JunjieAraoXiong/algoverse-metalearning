% ============================================================================
% RESEARCH NOTES: Financial RAG with Meta-Learning
% Author: Junjie Xiong
% Project: ICLR 2026 FinAI Workshop
%
% This document contains detailed notes on all concepts, experiments,
% decisions, and learnings from the research process.
% ============================================================================

\documentclass[11pt,a4paper]{article}

% Packages
\usepackage[utf8]{inputenc}
\usepackage[T1]{fontenc}
\usepackage[margin=1in]{geometry}
\usepackage{hyperref}
\usepackage{xcolor}
\usepackage{booktabs}
\usepackage{longtable}
\usepackage{colortbl}
\usepackage{multirow}
\usepackage{amsmath,amssymb}
\usepackage{graphicx}
\usepackage{tikz}
\usepackage{listings}
\usepackage{tcolorbox}
\usepackage{enumitem}
\usepackage{fancyhdr}
\usepackage{titlesec}

% Colors
\definecolor{conceptblue}{RGB}{0, 84, 147}
\definecolor{resultgreen}{RGB}{34, 139, 34}
\definecolor{warningyellow}{RGB}{255, 193, 7}
\definecolor{notegray}{RGB}{245, 245, 245}
\definecolor{codebg}{RGB}{248, 248, 248}

% Hyperref setup
\hypersetup{
    colorlinks=true,
    linkcolor=conceptblue,
    citecolor=conceptblue,
    urlcolor=conceptblue
}

% Code listings
\lstset{
    backgroundcolor=\color{codebg},
    basicstyle=\ttfamily\small,
    breaklines=true,
    frame=single,
    rulecolor=\color{gray!30}
}

% Custom boxes
\newtcolorbox{conceptbox}[1]{
    colback=conceptblue!5,
    colframe=conceptblue,
    fonttitle=\bfseries,
    title=#1
}

\newtcolorbox{resultbox}[1]{
    colback=resultgreen!5,
    colframe=resultgreen,
    fonttitle=\bfseries,
    title=#1
}

\newtcolorbox{warningbox}[1]{
    colback=warningyellow!10,
    colframe=warningyellow!80!black,
    fonttitle=\bfseries,
    title=#1
}

\newtcolorbox{notebox}{
    colback=notegray,
    colframe=gray!50,
    boxrule=0.5pt
}

% Header/Footer
\pagestyle{fancy}
\fancyhf{}
\fancyhead[L]{\textit{Research Notes: Financial RAG}}
\fancyhead[R]{\textit{Junjie Xiong}}
\fancyfoot[C]{\thepage}

% Section formatting
\titleformat{\section}{\Large\bfseries\color{conceptblue}}{\thesection}{1em}{}
\titleformat{\subsection}{\large\bfseries}{\thesubsection}{1em}{}

% ============================================================================
\begin{document}

% Title
\begin{center}
{\Huge\bfseries Research Notes}\\[0.5em]
{\LARGE Table-Aware RAG \& Learned Routing for Financial QA}\\[1em]
{\large Junjie Xiong}\\[0.5em]
{\normalsize Last Updated: January 2026}\\[0.5em]
{\normalsize Target: ICLR 2026 FinAI Workshop}
\end{center}

\vspace{1em}
\hrule
\vspace{1em}

% ============================================================================
% EXECUTIVE SUMMARY
% ============================================================================
\section*{Executive Summary}

\begin{resultbox}{TL;DR - What This Research Achieves}
\textbf{Problem:} RAG systems fail on financial questions requiring table data (27\% accuracy).

\textbf{Root Cause:} Standard PDF chunking fragments tables, destroying semantic structure.

\textbf{Solution:} Two contributions:
\begin{enumerate}
    \item \textbf{Table-Aware Chunking} (Docling + TableFormer): Preserves tables as atomic units $\rightarrow$ \textbf{+94\% improvement on metrics questions}
    \item \textbf{Learned Pipeline Router}: Predicts optimal retrieval strategy per question $\rightarrow$ \textbf{6$\times$ inference cost reduction}
\end{enumerate}

\textbf{Result:} 48.5\% $\rightarrow$ 60.4\% overall accuracy on FinanceBench (150 questions)
\end{resultbox}

\subsection*{Key Numbers to Remember}

\begin{tabular}{@{}lcc@{}}
\toprule
\textbf{Metric} & \textbf{Before} & \textbf{After} \\
\midrule
Overall Accuracy & 48.5\% & \textbf{60.4\%} \\
Metrics Questions & 26.7\% & \textbf{51.9\%} \\
Table Structure Accuracy & N/A & \textbf{97.9\%} \\
Inference Cost & 6$\times$ (run all) & \textbf{1$\times$} (routed) \\
\bottomrule
\end{tabular}

% ============================================================================
% READING PRIORITY GUIDE
% ============================================================================
\subsection*{Paper Reading Priority Guide}

Read these papers in order of priority for your research:

\begin{tabular}{@{}clp{6cm}c@{}}
\toprule
\textbf{Priority} & \textbf{Paper} & \textbf{Why You Need It} & \textbf{Time} \\
\midrule
\textcolor{red}{\textbf{P0}} & FinanceBench (2023) & Your benchmark - understand the task & 1 hr \\
\textcolor{red}{\textbf{P0}} & TableFormer (CVPR 2022) & Core of your table handling & 2 hrs \\
\textcolor{red}{\textbf{P0}} & Adaptive-RAG (NAACL 2024) & Closest related work to differentiate & 2 hrs \\
\midrule
\textcolor{orange}{\textbf{P1}} & RAG (NeurIPS 2020) & Foundational - establishes paradigm & 1.5 hrs \\
\textcolor{orange}{\textbf{P1}} & Self-RAG (ICLR 2024) & Key alternative approach & 2 hrs \\
\textcolor{orange}{\textbf{P1}} & MAML (ICML 2017) & Future work foundation & 2 hrs \\
\midrule
\textcolor{blue}{\textbf{P2}} & ProtoNets (NeurIPS 2017) & Simpler meta-learning alternative & 1 hr \\
\textcolor{blue}{\textbf{P2}} & CRAG (2024) & Post-hoc correction approach & 1 hr \\
\textcolor{blue}{\textbf{P2}} & RouteRAG (2025) & RL-based routing comparison & 1.5 hrs \\
\midrule
\textcolor{gray}{\textbf{P3}} & TAPAS, TURL & Table understanding background & 1 hr ea \\
\textcolor{gray}{\textbf{P3}} & Surveys (Gao, Hospedales) & Broader context & 2 hrs ea \\
\bottomrule
\end{tabular}

\vspace{1em}
\textbf{Recommended Order:} FinanceBench $\rightarrow$ TableFormer $\rightarrow$ Adaptive-RAG $\rightarrow$ RAG $\rightarrow$ Self-RAG $\rightarrow$ MAML $\rightarrow$ Others as needed.

\newpage

% ============================================================================
% GLOSSARY
% ============================================================================
\section*{Glossary of Key Terms}

\begin{longtable}{@{}p{4cm}p{10cm}@{}}
\toprule
\textbf{Term} & \textbf{Definition} \\
\midrule
\endhead

\textbf{RAG} & Retrieval-Augmented Generation. Combines LLM with external document retrieval to ground answers in evidence. \\
\midrule

\textbf{Chunking} & Process of splitting documents into smaller pieces for vector storage. Bad chunking = split tables = broken retrieval. \\
\midrule

\textbf{Embedding} & Dense vector representation of text. Similar texts have similar embeddings (high cosine similarity). \\
\midrule

\textbf{Top-k Retrieval} & Return the $k$ most similar chunks to a query based on embedding distance. \\
\midrule

\textbf{Semantic Search} & Retrieval based on meaning (embeddings) rather than exact keyword match. \\
\midrule

\textbf{BM25} & Classic keyword-based retrieval algorithm. Good for exact matches, bad for paraphrases. \\
\midrule

\textbf{Hybrid Retrieval} & Combines semantic (embedding) and lexical (BM25) retrieval. Usually via score fusion. \\
\midrule

\textbf{Reranking} & Second-stage scoring of retrieved documents using a more expensive model (cross-encoder). \\
\midrule

\textbf{Cross-Encoder} & Model that takes (query, document) pair and outputs relevance score. More accurate than bi-encoder but slower. \\
\midrule

\textbf{Oracle Labels} & Ground truth labels generated by running all options and picking the best. ``Cheating'' labels for training. \\
\midrule

\textbf{Meta-Learning} & Learning to learn. Training a model to quickly adapt to new tasks with few examples. \\
\midrule

\textbf{Few-Shot Learning} & Learning from very few examples (typically 1-20 per class). \\
\midrule

\textbf{Support Set} & In few-shot learning, the small labeled set used for adaptation. \\
\midrule

\textbf{Query Set} & In few-shot learning, the held-out set used to evaluate adapted model. \\
\midrule

\textbf{Prototype} & In ProtoNets, the mean embedding of all examples from a class. \\
\midrule

\textbf{Inner Loop} & In MAML, the task-specific adaptation step (gradient descent on support set). \\
\midrule

\textbf{Outer Loop} & In MAML, the meta-update step (update initialization based on post-adaptation performance). \\
\midrule

\textbf{FOMAML} & First-Order MAML. Ignores second-order gradients for faster training. \\
\midrule

\textbf{Reflection Tokens} & In Self-RAG, special tokens ([Retrieve], [IsRel], etc.) that guide generation. \\
\midrule

\textbf{Docling} & IBM's document processing tool. Uses TableFormer for table extraction. \\
\midrule

\textbf{TableFormer} & Transformer model for table structure recognition (97.9\% accuracy). \\
\midrule

\textbf{ChromaDB} & Open-source vector database for storing and querying embeddings. \\
\midrule

\textbf{FinanceBench} & Benchmark of 150 financial QA questions (50 metrics, 50 domain, 50 novel). \\
\midrule

\textbf{Metrics Questions} & Questions requiring numerical extraction from tables (hardest category). \\

\bottomrule
\end{longtable}

\newpage

% ============================================================================
% KEY EQUATIONS REFERENCE
% ============================================================================
\section*{Key Equations Reference Card}

\subsection*{RAG Fundamentals}

\textbf{Cosine Similarity} (used in semantic retrieval):
\begin{equation}
\text{sim}(q, d) = \frac{q \cdot d}{||q|| \cdot ||d||}
\end{equation}

\textbf{BM25 Score} (keyword retrieval):
\begin{equation}
\text{BM25}(q, d) = \sum_{t \in q} \text{IDF}(t) \cdot \frac{f(t, d) \cdot (k_1 + 1)}{f(t, d) + k_1 \cdot (1 - b + b \cdot \frac{|d|}{\text{avgdl}})}
\end{equation}

\textbf{Hybrid Score Fusion}:
\begin{equation}
\text{score}(d) = \alpha \cdot \text{semantic}(d) + (1-\alpha) \cdot \text{BM25}(d)
\end{equation}

\subsection*{Meta-Learning (MAML)}

\textbf{Inner Loop} (task adaptation):
\begin{equation}
\theta'_i = \theta - \alpha \nabla_\theta \mathcal{L}_{\mathcal{T}_i}(f_\theta)
\end{equation}

\textbf{Outer Loop} (meta-update):
\begin{equation}
\theta \leftarrow \theta - \beta \nabla_\theta \sum_{\mathcal{T}_i} \mathcal{L}_{\mathcal{T}_i}(f_{\theta'_i})
\end{equation}

\textbf{FOMAML Approximation}:
\begin{equation}
\nabla_\theta \mathcal{L}(f_{\theta'_i}) \approx \nabla_{\theta'_i} \mathcal{L}(f_{\theta'_i})
\end{equation}

\subsection*{Prototypical Networks}

\textbf{Class Prototype}:
\begin{equation}
c_k = \frac{1}{|S_k|} \sum_{(x_i, y_i) \in S_k} f_\phi(x_i)
\end{equation}

\textbf{Classification Probability}:
\begin{equation}
p(y=k | x) = \frac{\exp(-||f_\phi(x) - c_k||^2)}{\sum_{k'} \exp(-||f_\phi(x) - c_{k'}||^2)}
\end{equation}

\subsection*{TableFormer Loss}

\textbf{Combined Training Objective}:
\begin{equation}
\mathcal{L}_{total} = \mathcal{L}_{structure} + \lambda \mathcal{L}_{bbox}
\end{equation}

where $\mathcal{L}_{bbox} = \mathcal{L}_{L1} + \mathcal{L}_{GIoU}$

\subsection*{Evaluation Metric}

\textbf{Semantic Similarity} (our primary metric):
\begin{equation}
\text{score} = \frac{\text{embed}(\text{pred}) \cdot \text{embed}(\text{gold})}{||\text{embed}(\text{pred})|| \cdot ||\text{embed}(\text{gold})||}
\end{equation}

\newpage

% ============================================================================
% MASTER COMPARISON TABLE
% ============================================================================
\section*{Master Comparison: All RAG Approaches}

\begin{longtable}{@{}p{2.2cm}p{2.5cm}p{3cm}p{2.5cm}p{3cm}@{}}
\toprule
\textbf{Method} & \textbf{Routing Signal} & \textbf{Training} & \textbf{LLM Changes} & \textbf{Key Limitation} \\
\midrule
\endhead

\textbf{Na\"ive RAG} & None (fixed) & None & None & One-size-fits-all \\
\midrule

\textbf{Adaptive-RAG} & Query complexity & Auto-labels from performance & None & Misses domain-specific needs \\
\midrule

\textbf{Self-RAG} & Reflection tokens & Critic-generated data & Fine-tuning required & Expensive, model-specific \\
\midrule

\textbf{CRAG} & Post-hoc confidence & Fine-tuned T5 evaluator & None & Reactive, adds latency \\
\midrule

\textbf{RouteRAG} & RL policy & Reinforcement learning & Backbone LLM & Unstable training \\
\midrule

\rowcolor{resultgreen!20}
\textbf{Ours} & Domain features (21) & Supervised (oracle labels) & \textbf{None} & Needs oracle generation \\

\bottomrule
\end{longtable}

\vspace{1em}

\begin{resultbox}{Why Our Approach Wins for Finance}
\begin{enumerate}
    \item \textbf{Domain-aware}: 21 features capture financial patterns (years, metrics, companies)
    \item \textbf{Table-aware}: Docling + TableFormer preserves table structure (97.9\% accuracy)
    \item \textbf{LLM-agnostic}: Works with any off-the-shelf model
    \item \textbf{Stable training}: Supervised learning, no RL instability
    \item \textbf{Interpretable}: Can inspect which features drive routing decisions
\end{enumerate}
\end{resultbox}

\newpage

\tableofcontents
\newpage

% ============================================================================
% PART 1: PROJECT OVERVIEW
% ============================================================================
\part{Project Overview}

\section{Research Goal}

\begin{conceptbox}{The Central Question}
Can we build a RAG system for financial documents that:
\begin{enumerate}
    \item Accurately extracts numerical data from tables
    \item Automatically selects the best retrieval strategy per question
    \item Generalizes across financial domains (and potentially other domains)
\end{enumerate}
\end{conceptbox}

\subsection{Why This Matters}

Financial AI assistants must answer precise questions like:
\begin{itemize}
    \item ``What was Apple's FY2023 operating margin?''
    \item ``Calculate the year-over-year revenue growth for Microsoft.''
    \item ``What risks does the company face related to supply chain?''
\end{itemize}

Current RAG systems fail badly on the first two types (numerical/table questions) because:
\begin{enumerate}
    \item \textbf{Tables get fragmented} during PDF chunking
    \item \textbf{Static pipelines} don't adapt to question type
    \item \textbf{No verification} catches hallucinated numbers
\end{enumerate}

\subsection{Two Contributions}

\begin{resultbox}{Contribution 1: Table-Aware Chunking}
Using Docling + TableFormer for PDF ingestion:
\begin{itemize}
    \item Tables preserved as atomic units (never split)
    \item 97.9\% table structure recognition accuracy
    \item Metrics questions: 27\% $\rightarrow$ 52\% (+94\% relative improvement)
\end{itemize}
\end{resultbox}

\begin{resultbox}{Contribution 2: Learned Pipeline Routing}
Training a classifier to predict optimal retrieval configuration:
\begin{itemize}
    \item 21 hand-crafted question features
    \item Predicts best pipeline without running all options
    \item 6$\times$ inference cost reduction
\end{itemize}
\end{resultbox}

\subsection{Timeline \& Venue}

\begin{itemize}
    \item \textbf{Target:} ICLR 2026 Workshop on Financial AI
    \item \textbf{Timeline:} 2+ weeks
    \item \textbf{Environment:} Together AI GPU cluster (SLURM)
\end{itemize}

% ============================================================================
% PART 2: BACKGROUND CONCEPTS
% ============================================================================
\newpage
\part{Background Concepts}

\section{Retrieval-Augmented Generation (RAG)}

\subsection{What is RAG?}

RAG combines two types of knowledge:
\begin{enumerate}
    \item \textbf{Parametric knowledge}: What the LLM learned during training (stored in weights)
    \item \textbf{Non-parametric knowledge}: External documents retrieved at inference time
\end{enumerate}

\begin{conceptbox}{RAG Pipeline}
\begin{verbatim}
User Question
     |
     v
[RETRIEVER] ---> Query vector database
     |
     v
Retrieved Documents (context)
     |
     v
[GENERATOR] ---> LLM generates answer using context
     |
     v
Final Answer
\end{verbatim}
\end{conceptbox}

\subsection{Why RAG?}

\begin{itemize}
    \item \textbf{Reduces hallucinations}: Answers grounded in retrieved evidence
    \item \textbf{Updatable knowledge}: Change documents, not model weights
    \item \textbf{Source citations}: Can point to where answer came from
    \item \textbf{Domain adaptation}: Add domain documents without fine-tuning
\end{itemize}

\subsection{RAG Evolution}

The field has evolved in stages:

\begin{tabular}{@{}lll@{}}
\toprule
\textbf{Stage} & \textbf{Approach} & \textbf{Limitation} \\
\midrule
Na\"ive RAG & Single retrieval, static $k$ & One-size-fits-all \\
Advanced RAG & Better retrievers, reranking & Still static per query \\
Adaptive RAG & Dynamic per-query decisions & \textbf{Current frontier} \\
\bottomrule
\end{tabular}

\vspace{1em}
\textbf{Our work is Adaptive RAG} -- we learn which strategy to use per question.

% ============================================================================
\section{Meta-Learning (Learning to Learn)}

\subsection{Core Idea}

\begin{conceptbox}{Meta-Learning vs Regular Learning}
\textbf{Regular learning:} ``Learn to solve this specific task well.''

\textbf{Meta-learning:} ``Learn to learn -- find a starting point that lets you quickly adapt to ANY new task.''
\end{conceptbox}

\subsection{Why It Matters for Our Project}

Our router is doing meta-learning at the pipeline level:
\begin{itemize}
    \item It learns \textit{which tool to use}, not \textit{how to answer}
    \item It generalizes patterns: ``questions with years + metrics $\rightarrow$ hybrid retrieval''
    \item It could adapt to new domains with few examples (future work)
\end{itemize}

\subsection{MAML (Model-Agnostic Meta-Learning)}

\subsubsection{The Algorithm}

MAML (Finn et al., 2017) learns a good \textit{initialization} that can quickly adapt to new tasks:

\begin{conceptbox}{MAML Two-Loop Structure}
\begin{verbatim}
OUTER LOOP (meta-learning):
    For each iteration:
        Sample batch of tasks {T1, T2, ..., Tn}

        INNER LOOP (task adaptation):
            For each task Ti:
                1. Copy parameters: theta' = theta
                2. Take k gradient steps on Ti's data
                3. Evaluate adapted theta' on Ti's test data

        META UPDATE:
            Update theta based on adapted performance
            theta = theta - beta * gradient(sum of losses)
\end{verbatim}
\end{conceptbox}

\subsubsection{Visual Intuition}

Imagine parameter space where different tasks have different optimal solutions:

\begin{verbatim}
                    Parameter Space

        * Task A optimum

                theta_0 (MAML initialization)
               /|\
              / | \
             /  |  \     <-- just a few gradient steps!
            v   v   v

    * Task B    * Task C    * Task D
    optimum     optimum     optimum
\end{verbatim}

MAML finds $\theta_0$ -- a central position from which you can quickly reach any task's optimum.

\subsubsection{Why ``Model-Agnostic''?}

MAML works with ANY model trained with gradient descent:
\begin{itemize}
    \item Neural networks
    \item CNNs for images
    \item Transformers for NLP
    \item Our pipeline router
\end{itemize}

\subsubsection{MAML for Our Router (Future Work)}

\textbf{Tasks = Different domains:}
\begin{itemize}
    \item Task 1: Finance questions (FinanceBench)
    \item Task 2: Medical questions (PubMedQA)
    \item Task 3: Legal questions (CUAD)
\end{itemize}

\textbf{Goal:} Learn router initialization that quickly adapts to any domain.

\begin{lstlisting}[language=Python]
# Meta-training
tasks = [FinanceRouterTask, MedicalRouterTask, LegalRouterTask]
theta_maml = maml_train(router_model, tasks)

# At deployment: New domain (e.g., Scientific papers)
new_task = ScientificRouterTask(support_size=20)  # Only 20 examples!

# Quick adaptation (just 5 gradient steps)
theta_adapted = theta_maml
for step in range(5):
    loss = compute_loss(theta_adapted, new_task.support)
    theta_adapted = theta_adapted - alpha * gradient(loss)

# Now router works for scientific domain!
\end{lstlisting}

\subsection{Prototypical Networks}

Alternative to MAML that's simpler (no inner loop):

\begin{conceptbox}{ProtoNets Idea}
\begin{enumerate}
    \item Learn an embedding space where similar items cluster together
    \item For each class, compute ``prototype'' = mean of class embeddings
    \item At inference: classify by nearest prototype
\end{enumerate}
\end{conceptbox}

For our router, this could mean:
\begin{lstlisting}[language=Python]
# Compute prototype for each pipeline
prototypes = {
    "semantic_k5": mean(embeddings where semantic_k5 won),
    "hybrid_k10": mean(embeddings where hybrid_k10 won),
    ...
}

# Route by nearest prototype
def route(question):
    q_embed = embed(question)
    return argmin(distance(q_embed, proto) for proto in prototypes)
\end{lstlisting}

\subsection{Comparison of Meta-Learning Approaches}

\begin{longtable}{@{}p{3cm}p{4cm}p{4cm}p{3cm}@{}}
\toprule
\textbf{Approach} & \textbf{What It Learns} & \textbf{Test Time Requirement} & \textbf{Complexity} \\
\midrule
\endhead
Regular Training & Task-specific params & Full retraining for new task & Low \\
Transfer Learning & Pretrained features & Fine-tuning (100s examples) & Medium \\
MAML & Optimal initialization & Few-shot adaptation (5-20 ex) & High \\
ProtoNets & Metric space & Just class prototypes & Medium \\
In-Context (GPT-3) & Implicit from pretraining & Examples in prompt & Low \\
\bottomrule
\end{longtable}

% ============================================================================
\section{The Three Scripts We Created}

\subsection{Script 1: generate\_oracle\_labels.py}

\subsubsection{Purpose}
Runs EVERY question through EVERY pipeline configuration and records which one worked best.

\subsubsection{What It Does}

\begin{lstlisting}[language=Python]
Question: "What was Apple's revenue in FY2023?"

Pipeline Results:
  semantic_k5:              0.72 similarity
  semantic_k10:             0.75 similarity
  semantic_k20:             0.74 similarity
  hybrid_filter_rerank_k5:  0.81 similarity  <- BEST
  hybrid_filter_rerank_k10: 0.79 similarity
  hybrid_filter_rerank_k20: 0.78 similarity

Oracle Label: hybrid_filter_rerank, k=5
\end{lstlisting}

\subsubsection{Why It Matters}

This creates \textbf{ground truth} for training. Without knowing which pipeline \textit{actually} works best for each question, we can't train a router to predict it.

\begin{warningbox}{Key Insight}
Oracle labels are ``cheating'' labels -- they require running all pipelines (expensive). The goal is to train a model that predicts these labels \textit{without} running all pipelines, using only the question text.
\end{warningbox}

\subsubsection{Output Format}

\begin{lstlisting}[language=Python]
# data/oracle_labels.json
{
    "question_id": {
        "question": "...",
        "question_type": "metrics-generated",
        "best_pipeline": "hybrid_filter_rerank",
        "best_top_k": 10,
        "best_score": 0.87,
        "all_scores": {"semantic_5": 0.72, "hybrid_10": 0.87, ...}
    }
}
\end{lstlisting}

\subsection{Script 2: train\_router.py}

\subsubsection{Purpose}
Trains a classifier to predict the best pipeline from question features alone (no retrieval needed).

\subsubsection{The 21 Features}

\begin{longtable}{@{}p{2cm}p{4cm}p{6cm}@{}}
\toprule
\textbf{Category} & \textbf{Feature} & \textbf{Intuition} \\
\midrule
\endhead
\multirow{4}{*}{Temporal}
    & has\_year & Question references specific year \\
    & has\_quarter & References Q1-Q4 \\
    & has\_fiscal\_indicator & Contains ``FY'' or ``fiscal'' \\
    & year\_count & How many years mentioned \\
\midrule
\multirow{3}{*}{Entity}
    & has\_company\_indicator & Company name present \\
    & capitalized\_word\_count & Potential named entities \\
    & has\_metric\_name & Revenue, EBITDA, etc. \\
\midrule
\multirow{6}{*}{Structure}
    & is\_what & Starts with ``What'' \\
    & is\_how & Starts with ``How'' \\
    & is\_why & Starts with ``Why'' \\
    & is\_yes\_no & Binary answer expected \\
    & expects\_number & Numerical output expected \\
    & expects\_explanation & Explanation needed \\
\midrule
\multirow{3}{*}{Stats}
    & word\_count & Total words \\
    & char\_count & Total characters \\
    & avg\_word\_length & Average word length \\
\midrule
\multirow{3}{*}{Domain}
    & finance\_density & Financial keyword density \\
    & medical\_density & Medical keyword density \\
    & legal\_density & Legal keyword density \\
\midrule
\multirow{2}{*}{Complexity}
    & needs\_reasoning & Multi-step logic required \\
    & multi\_part\_question & Multiple sub-questions \\
\bottomrule
\end{longtable}

\subsubsection{Training Process}

\begin{enumerate}
    \item Load oracle labels from JSON
    \item Extract 21 features for all 150 questions
    \item Train sklearn classifier (RandomForest or XGBoost)
    \item 5-fold cross-validation
    \item Save model to \texttt{models/pipeline\_router.pkl}
\end{enumerate}

\subsubsection{Why This Matters}

\begin{resultbox}{Inference Speedup}
\textbf{Before (naive):} Run all 6 pipelines $\rightarrow$ Pick best $\rightarrow$ 6$\times$ cost

\textbf{After (learned):} Extract features $\rightarrow$ Predict best $\rightarrow$ 1$\times$ cost
\end{resultbox}

\subsection{Script 3: error\_analysis.py}

\subsubsection{Purpose}
Analyzes questions where the system still fails and categorizes them.

\subsubsection{Error Categories}

\begin{tabular}{@{}p{3cm}p{4cm}p{5cm}@{}}
\toprule
\textbf{Category} & \textbf{Description} & \textbf{Example} \\
\midrule
Missing Table & Table not in any chunk & Revenue table fragmented \\
Wrong Period & Retrieved wrong year/quarter & Got 2022 instead of 2023 \\
Wrong Company & Retrieved different company & Mixed up Apple/Microsoft \\
Calculation Error & LLM math wrong & Computed ratio incorrectly \\
Format Mismatch & Answer format differs & ``\$1.2B'' vs ``1200 million'' \\
Hallucination & Number not in context & Made up a value \\
\bottomrule
\end{tabular}

\subsubsection{Why This Matters}

\begin{enumerate}
    \item \textbf{For the paper}: Analysis section needs this -- reviewers want to know WHY you're still at 60\% instead of 100\%
    \item \textbf{For future work}: Identifies what to fix next
\end{enumerate}

\subsection{How The Three Scripts Connect}

\begin{verbatim}
+-------------------------------------------------------------+
|                    TRAINING PHASE                           |
|                                                             |
|   FinanceBench        generate_oracle        train_router   |
|   150 questions  ---> labels.py         ---> .py           |
|                       (900 API calls)        (5-fold CV)    |
|                              |                    |         |
|                              v                    v         |
|                    oracle_labels.json    router_rf.joblib   |
+-------------------------------------------------------------+

+-------------------------------------------------------------+
|                    INFERENCE PHASE                          |
|                                                             |
|   New Question --> Extract 21 --> Router --> Best Pipeline  |
|                    Features       Predicts    (1 API call)  |
|                    (<1ms)         (<1ms)                    |
+-------------------------------------------------------------+

+-------------------------------------------------------------+
|                    ANALYSIS PHASE                           |
|                                                             |
|   Evaluation     error_analysis    docs/error_analysis.md   |
|   Results    --> .py          --> (for paper section)       |
+-------------------------------------------------------------+
\end{verbatim}

% ============================================================================
% PART 3: EXPERIMENT RESULTS
% ============================================================================
\newpage
\part{Experiment Results}

\section{The Breakthrough Finding}

\begin{resultbox}{Key Discovery - January 10, 2026}
\textbf{Table handling was the bottleneck, NOT the model!}

\begin{itemize}
    \item Old chunking (PyPDF): 28\% metrics
    \item New chunking (Docling): \textbf{52\% metrics} (+94\% improvement!)
    \item Overall: 48.5\% $\rightarrow$ \textbf{60.4\%} (+24\% improvement!)
\end{itemize}
\end{resultbox}

\section{Complete Results History}

\subsection{Baseline Experiments (PyPDF Chunking)}

\subsubsection{Run 1: Initial Baseline}

\begin{tabular}{@{}ll@{}}
\toprule
\textbf{Configuration} & \textbf{Value} \\
\midrule
Date & 2026-01-09 22:25 \\
Model & Llama 3.1 70B Turbo \\
Pipeline & Semantic \\
Top-k & 10 \\
Embeddings & OpenAI text-embedding-3-large \\
\bottomrule
\end{tabular}

\vspace{0.5em}
\textbf{Results:}
\begin{tabular}{@{}lc@{}}
\toprule
\textbf{Metric} & \textbf{Score} \\
\midrule
Overall & 0.485 (48.5\%) \\
Domain-relevant (n=50) & 0.628 (62.8\%) \\
Novel-generated (n=50) & 0.560 (56.0\%) \\
Metrics-generated (n=50) & 0.267 (26.7\%) \\
Error Rate & 0.00\% \\
Retrieval Latency & 340ms \\
\bottomrule
\end{tabular}

\textbf{Observations:}
\begin{itemize}
    \item Pipeline is stable (0\% errors)
    \item Domain-relevant questions perform well (62.8\%)
    \item \textbf{Metrics-generated questions are broken (26.7\%)} -- need precise table extraction
\end{itemize}

\subsection{Top-k Ablation}

\begin{tabular}{@{}lccccc@{}}
\toprule
\textbf{Top-k} & \textbf{Overall} & \textbf{Domain} & \textbf{Novel} & \textbf{Metrics} & \textbf{$\Delta$ vs k=10} \\
\midrule
10 & 0.485 & 0.628 & 0.560 & 0.267 & -- \\
15 & 0.486 & 0.634 & 0.557 & 0.267 & +0.1\% \\
20 & 0.489 & 0.639 & 0.558 & 0.270 & +0.8\% \\
30 & 0.493 & 0.637 & 0.565 & 0.275 & +1.6\% \\
\bottomrule
\end{tabular}

\begin{warningbox}{Learning}
More context helps slightly, but diminishing returns. k=10 $\rightarrow$ k=30 = only +1.6\%.
\end{warningbox}

\subsection{Pipeline Ablation}

\begin{tabular}{@{}lcccc@{}}
\toprule
\textbf{Pipeline} & \textbf{Overall} & \textbf{Metrics} & \textbf{Latency} & \textbf{$\Delta$} \\
\midrule
Semantic k=10 & 0.485 & 0.267 & 340ms & -- \\
Hybrid k=10 & 0.491 & 0.271 & 3330ms & +1.2\% \\
Hybrid k=20 & 0.496 & 0.278 & 2646ms & +2.3\% \\
\bottomrule
\end{tabular}

\begin{warningbox}{Learning}
Reranking helps slightly (+2.3\%) but is 8-10$\times$ slower.
\end{warningbox}

\subsection{Model Ablation}

\begin{tabular}{@{}lcccc@{}}
\toprule
\textbf{Model} & \textbf{Overall} & \textbf{Domain} & \textbf{Novel} & \textbf{Metrics} \\
\midrule
Llama 3.1 70B & 0.485 & 0.628 & 0.560 & 0.267 \\
Llama 3.3 70B & 0.477 & 0.621 & 0.550 & 0.260 \\
Qwen 2.5 72B & 0.491 & 0.635 & 0.562 & 0.275 \\
GPT-4o-mini & 0.490 & 0.632 & 0.560 & 0.279 \\
\midrule
\textit{Std. Dev.} & 0.006 & 0.006 & 0.005 & 0.008 \\
\bottomrule
\end{tabular}

\begin{resultbox}{Critical Insight}
\textbf{All models hit the same $\sim$28\% ceiling on metrics questions.}

This proves the bottleneck is in retrieval/chunking, NOT the LLM. Switching models doesn't solve the table extraction problem.
\end{resultbox}

\subsection{Final Leaderboard (Pre-Docling)}

\begin{tabular}{@{}clcc@{}}
\toprule
\textbf{Rank} & \textbf{Configuration} & \textbf{Overall} & \textbf{Metrics} \\
\midrule
1 & Hybrid k=20 + Llama 3.1 & 0.496 & 0.278 \\
2 & Semantic k=30 + Llama 3.1 & 0.493 & 0.275 \\
3 & Semantic k=10 + Qwen 2.5 & 0.491 & 0.275 \\
4 & Hybrid k=10 + Llama 3.1 & 0.491 & 0.271 \\
5 & Semantic k=10 + GPT-4o-mini & 0.490 & 0.279 \\
6 & Semantic k=20 + Llama 3.1 & 0.489 & 0.270 \\
7 & Semantic k=15 + Llama 3.1 & 0.486 & 0.267 \\
8 & Semantic k=10 + Llama 3.1 & 0.485 & 0.267 \\
9 & Semantic k=10 + Llama 3.3 & 0.477 & 0.260 \\
\bottomrule
\end{tabular}

\subsection{Docling + TableFormer Results (Breakthrough)}

\begin{tabular}{@{}lcccc@{}}
\toprule
\textbf{Configuration} & \textbf{Overall} & \textbf{Metrics} & \textbf{Domain} & \textbf{Novel} \\
\midrule
PyPDF + Llama 70B (k=10) & 0.485 & 0.267 & 0.628 & 0.560 \\
PyPDF + Llama 70B (k=20) & 0.489 & 0.270 & 0.639 & 0.558 \\
\rowcolor{resultgreen!20}
\textbf{Docling + GPT-4o-mini (k=5)} & \textbf{0.604} & \textbf{0.519} & \textbf{0.689} & \textbf{0.604} \\
\bottomrule
\end{tabular}

\subsection{Improvement Analysis}

\begin{tabular}{@{}lcccc@{}}
\toprule
\textbf{Metric} & \textbf{Old} & \textbf{New} & \textbf{Absolute $\Delta$} & \textbf{Relative $\Delta$} \\
\midrule
Overall & 0.485 & 0.604 & +0.119 & \textbf{+24.5\%} \\
Metrics-generated & 0.267 & 0.519 & +0.252 & \textbf{+94.4\%} \\
Domain-relevant & 0.628 & 0.689 & +0.061 & +9.7\% \\
Novel-generated & 0.560 & 0.604 & +0.044 & +7.9\% \\
\bottomrule
\end{tabular}

\subsection{Docling Ingestion Statistics}

\begin{tabular}{@{}ll@{}}
\toprule
\textbf{Metric} & \textbf{Value} \\
\midrule
PDFs Processed & 265/265 (100\%) \\
Total Chunks & 89,566 \\
Table Accuracy & 97.9\% (TableFormer) \\
ChromaDB Location & chroma\_docling/ \\
\bottomrule
\end{tabular}

% ============================================================================
% PART 4: RELATED WORK - COMPREHENSIVE LITERATURE REVIEW
% ============================================================================
\newpage
\part{Related Work \& Literature}

This part provides detailed summaries of all key papers relevant to our research. Reading these papers is essential for understanding the research landscape.

% ----------------------------------------------------------------------------
\section{Foundational Papers}
% ----------------------------------------------------------------------------

\subsection{RAG: Retrieval-Augmented Generation (Lewis et al., 2020)}

\begin{conceptbox}{Paper Details}
\textbf{Title:} Retrieval-Augmented Generation for Knowledge-Intensive NLP Tasks\\
\textbf{Authors:} Patrick Lewis, Ethan Perez, Aleksandra Piktus, Fabio Petroni, Vladimir Karpukhin, et al.\\
\textbf{Venue:} NeurIPS 2020\\
\textbf{arXiv:} \url{https://arxiv.org/abs/2005.11401}
\end{conceptbox}

\subsubsection{Core Contribution}

This is the foundational paper that established the RAG paradigm. The key insight is to combine \textbf{parametric memory} (LLM weights) with \textbf{non-parametric memory} (retrieved documents).

\subsubsection{Technical Approach}

RAG combines a pre-trained seq2seq model with a neural retriever:
\begin{enumerate}
    \item \textbf{Retriever:} Dense Passage Retrieval (DPR) to find relevant documents
    \item \textbf{Generator:} BART fine-tuned to generate answers conditioned on retrieved docs
\end{enumerate}

Two formulations are compared:
\begin{itemize}
    \item \textbf{RAG-Sequence:} Same documents for entire generation
    \item \textbf{RAG-Token:} Can use different documents per generated token
\end{itemize}

\subsubsection{Key Results}

\begin{itemize}
    \item State-of-the-art on 3 open-domain QA tasks
    \item RAG generates more specific, diverse, and factual language than parametric-only models
    \item Outperforms task-specific retrieve-and-extract architectures
\end{itemize}

\subsubsection{Why It Matters for Us}

This paper established the paradigm we're building on. Every RAG system since, including ours, follows this basic structure of retrieve-then-generate. Our contribution is making the retrieval step adaptive rather than fixed.

% ----------------------------------------------------------------------------
\subsection{MAML: Model-Agnostic Meta-Learning (Finn et al., 2017)}

\begin{conceptbox}{Paper Details}
\textbf{Title:} Model-Agnostic Meta-Learning for Fast Adaptation of Deep Networks\\
\textbf{Authors:} Chelsea Finn, Pieter Abbeel, Sergey Levine\\
\textbf{Venue:} ICML 2017\\
\textbf{arXiv:} \url{https://arxiv.org/abs/1703.03400}\\
\textbf{Code:} \url{https://github.com/cbfinn/maml}
\end{conceptbox}

\subsubsection{Core Contribution}

MAML proposes an algorithm for meta-learning that is \textbf{model-agnostic} -- it works with any model trained with gradient descent. The key idea is to learn an \textbf{initialization} from which the model can quickly adapt to new tasks.

\subsubsection{Technical Approach}

The algorithm has a distinctive \textbf{bi-level optimization}:

\textbf{Inner Loop (Task Adaptation):}
\begin{equation}
\theta'_i = \theta - \alpha \nabla_\theta \mathcal{L}_{T_i}(f_\theta)
\end{equation}

\textbf{Outer Loop (Meta-Update):}
\begin{equation}
\theta \leftarrow \theta - \beta \nabla_\theta \sum_{T_i \sim p(\mathcal{T})} \mathcal{L}_{T_i}(f_{\theta'_i})
\end{equation}

The crucial insight is that the outer loop optimizes for \textbf{post-adaptation performance}, not initial performance. This trains the model to be easy to fine-tune.

\subsubsection{Key Results}

\begin{itemize}
    \item State-of-the-art on Omniglot (1-shot: 98.7\%, 5-shot: 99.9\%)
    \item Strong results on Mini-ImageNet few-shot classification
    \item Works for classification, regression, AND reinforcement learning
\end{itemize}

\subsubsection{Why It Matters for Us}

MAML provides the theoretical foundation for training our router to generalize across domains. In future work, we could use MAML to train a router that:
\begin{itemize}
    \item Pre-trains on Finance, Medical, Legal domains
    \item Quickly adapts to new domains (e.g., Scientific) with only 20 examples
\end{itemize}

% ----------------------------------------------------------------------------
\subsection{Prototypical Networks (Snell et al., 2017)}

\begin{conceptbox}{Paper Details}
\textbf{Title:} Prototypical Networks for Few-shot Learning\\
\textbf{Authors:} Jake Snell, Kevin Swersky, Richard S. Zemel\\
\textbf{Venue:} NeurIPS 2017\\
\textbf{arXiv:} \url{https://arxiv.org/abs/1703.05175}\\
\textbf{Code:} \url{https://github.com/jakesnell/prototypical-networks}
\end{conceptbox}

\subsubsection{Core Contribution}

ProtoNets propose a simpler alternative to MAML: learn a \textbf{metric space} where classification is done by finding the \textbf{nearest class prototype} (mean of support examples).

\subsubsection{Technical Approach}

\begin{enumerate}
    \item \textbf{Embed} all support examples using a neural network $f_\phi$
    \item \textbf{Compute prototypes}: $c_k = \frac{1}{|S_k|} \sum_{(x_i, y_i) \in S_k} f_\phi(x_i)$
    \item \textbf{Classify} query by nearest prototype (Euclidean distance)
\end{enumerate}

\subsubsection{Key Finding: Distance Function Matters}

\begin{warningbox}{Critical Insight}
\textbf{Euclidean distance significantly outperforms cosine similarity!}

This is counterintuitive since most NLP work uses cosine similarity for embeddings. The paper shows that for few-shot classification, Euclidean distance in the learned space works better.
\end{warningbox}

\subsubsection{Why It Matters for Us}

For our router, ProtoNets provide a simpler baseline than MAML:
\begin{lstlisting}[language=Python]
# Router as ProtoNet
prototypes = {
    "semantic_k5": mean(embeddings where semantic_k5 won),
    "hybrid_k10": mean(embeddings where hybrid_k10 won),
    ...
}

def route(question):
    q_embed = embed(question)
    return argmin(euclidean(q_embed, proto) for proto in prototypes)
\end{lstlisting}

% ----------------------------------------------------------------------------
\subsection{Meta-Learning Survey (Hospedales et al., 2021)}

\begin{conceptbox}{Paper Details}
\textbf{Title:} Meta-Learning in Neural Networks: A Survey\\
\textbf{Authors:} Timothy Hospedales, Antreas Antoniou, Paul Micaelli, Amos Storkey\\
\textbf{Venue:} IEEE TPAMI 2022 (arXiv 2020)\\
\textbf{arXiv:} \url{https://arxiv.org/abs/2004.05439}
\end{conceptbox}

\subsubsection{Key Taxonomy}

The survey proposes a comprehensive taxonomy:

\begin{tabular}{@{}lp{8cm}@{}}
\toprule
\textbf{Category} & \textbf{Examples} \\
\midrule
Metric-based & ProtoNets, Matching Networks, Relation Networks \\
Optimization-based & MAML, Reptile, Meta-SGD \\
Model-based & MANN, Neural Turing Machines \\
Black-box & Hypernetworks, learned optimizers \\
\bottomrule
\end{tabular}

\subsubsection{Applications Covered}

\begin{itemize}
    \item Few-shot learning
    \item Reinforcement learning
    \item Neural architecture search
    \item Hyperparameter optimization
    \item Continual learning
\end{itemize}

\subsubsection{Why It Matters for Us}

This survey provides the background context for positioning our work. Our router combines elements of:
\begin{itemize}
    \item \textbf{Metric-based} (learned features for routing)
    \item \textbf{Optimization-based} (potential MAML extension)
\end{itemize}

% ----------------------------------------------------------------------------
\section{Adaptive RAG Papers}
% ----------------------------------------------------------------------------

\subsection{Adaptive-RAG (Jeong et al., 2024)}

\begin{conceptbox}{Paper Details}
\textbf{Title:} Adaptive-RAG: Learning to Adapt Retrieval-Augmented Large Language Models through Question Complexity\\
\textbf{Authors:} Soyeong Jeong, Jinheon Baek, Sukmin Cho, Sung Ju Hwang, Jong Park\\
\textbf{Venue:} NAACL 2024 (pages 7036-7050)\\
\textbf{arXiv:} \url{https://arxiv.org/abs/2403.14403}\\
\textbf{Code:} \url{https://github.com/starsuzi/Adaptive-RAG}
\end{conceptbox}

\subsubsection{Core Contribution}

Adaptive-RAG introduces a \textbf{query-complexity classifier} that routes questions to one of three strategies:

\begin{enumerate}
    \item \textbf{No retrieval} -- Simple facts the LLM already knows
    \item \textbf{Single-step retrieval} -- Standard RAG for moderate queries
    \item \textbf{Iterative/multi-step retrieval} -- Complex multi-hop questions
\end{enumerate}

\subsubsection{Technical Approach}

\begin{itemize}
    \item \textbf{Classifier:} Small LM trained to predict complexity
    \item \textbf{Labels:} Automatically generated from model performance on different strategies
    \item \textbf{Training signal:} If no-retrieval beats retrieval, label = ``simple''
\end{itemize}

\subsubsection{Key Results}

\begin{itemize}
    \item Outperforms static single-step and iterative baselines
    \item Reduces unnecessary retrieval for simple queries
    \item Improves accuracy on complex multi-hop queries
\end{itemize}

\subsubsection{How We Differ}

\begin{resultbox}{Our Differentiation}
\textbf{Adaptive-RAG routes by complexity}; \textbf{we route by domain-specific features}.

A ``simple'' revenue question still needs table retrieval -- complexity alone misses this. Our 21 features capture:
\begin{itemize}
    \item Temporal signals (has\_year, has\_quarter)
    \item Financial signals (has\_metric\_name, finance\_density)
    \item Structure (expects\_number, needs\_reasoning)
\end{itemize}
\end{resultbox}

% ----------------------------------------------------------------------------
\subsection{Self-RAG (Asai et al., 2023)}

\begin{conceptbox}{Paper Details}
\textbf{Title:} Self-RAG: Learning to Retrieve, Generate, and Critique through Self-Reflection\\
\textbf{Authors:} Akari Asai, Zeqiu Wu, Yizhong Wang, Avirup Sil, Hannaneh Hajishirzi\\
\textbf{Venue:} ICLR 2024 (Oral presentation, top 1\%)\\
\textbf{arXiv:} \url{https://arxiv.org/abs/2310.11511}\\
\textbf{Code:} \url{https://github.com/AkariAsai/self-rag}
\end{conceptbox}

\subsubsection{Core Contribution}

Self-RAG trains an LM to generate text with special \textbf{reflection tokens}:
\begin{itemize}
    \item \texttt{[Retrieve]} -- Should I retrieve now?
    \item \texttt{[IsRel]} -- Is retrieved document relevant?
    \item \texttt{[IsSup]} -- Is generation supported by evidence?
    \item \texttt{[IsUse]} -- Is response useful?
\end{itemize}

\subsubsection{Technical Approach}

\begin{enumerate}
    \item Train a \textbf{critic model} to insert reflection tokens into training data
    \item Fine-tune LM on this data to generate tokens + reflections
    \item At inference, model decides when to retrieve and self-critiques
\end{enumerate}

\subsubsection{Key Results}

\begin{itemize}
    \item Outperforms ChatGPT on Open-domain QA, reasoning, and fact verification
    \item 7B and 13B versions beat retrieval-augmented Llama2-chat
    \item Significant gains in factuality and citation accuracy
\end{itemize}

\subsubsection{How We Differ}

\begin{warningbox}{Key Difference}
\textbf{Self-RAG modifies the LLM}; \textbf{we keep the LLM unchanged}.

Self-RAG requires fine-tuning the generator to emit reflection tokens. Our approach:
\begin{itemize}
    \item Uses any off-the-shelf LLM
    \item Routes at the pipeline level before generation
    \item No additional token vocabulary or training
\end{itemize}
\end{warningbox}

% ----------------------------------------------------------------------------
\subsection{CRAG: Corrective RAG (Yan et al., 2024)}

\begin{conceptbox}{Paper Details}
\textbf{Title:} Corrective Retrieval Augmented Generation\\
\textbf{Authors:} Shi-Qi Yan, Jia-Chen Gu, Yun Zhu, Zhen-Hua Ling\\
\textbf{Venue:} arXiv 2024\\
\textbf{arXiv:} \url{https://arxiv.org/abs/2401.15884}\\
\textbf{Code:} \url{https://github.com/HuskyInSalt/CRAG}
\end{conceptbox}

\subsubsection{Core Contribution}

CRAG evaluates retrieval quality \textbf{post-hoc} and takes corrective actions when confidence is low:

\begin{enumerate}
    \item \textbf{Correct} -- High confidence, use retrieved docs directly
    \item \textbf{Incorrect} -- Low confidence, fall back to web search
    \item \textbf{Ambiguous} -- Medium confidence, combine both sources
\end{enumerate}

\subsubsection{Technical Approach}

\begin{itemize}
    \item \textbf{Retrieval Evaluator:} Fine-tuned T5-large (0.77B params) -- much lighter than Self-RAG's critic
    \item \textbf{Knowledge Refinement:} Decompose-then-recompose algorithm to filter irrelevant info
    \item \textbf{Web Augmentation:} Uses large-scale web search as fallback
\end{itemize}

\subsubsection{Key Results}

\begin{itemize}
    \item Significantly improves upon standard RAG
    \item Outperforms Self-RAG on PopQA, Biography, Pub Health, Arc-Challenge
    \item \textbf{Plug-and-play}: Can be added to any RAG system
\end{itemize}

\subsubsection{How We Differ}

\begin{resultbox}{Our Differentiation}
\textbf{CRAG corrects bad retrievals after they happen}; \textbf{we prevent them by choosing the right pipeline upfront}.

CRAG's approach:
\begin{itemize}
    \item Retrieve $\rightarrow$ Evaluate $\rightarrow$ Correct (if needed)
    \item Adds latency with post-hoc evaluation
    \item Web search fallback may not help domain-specific questions
\end{itemize}

Our approach:
\begin{itemize}
    \item Analyze question $\rightarrow$ Route to best pipeline $\rightarrow$ Retrieve once
    \item No post-hoc correction needed
    \item Domain-specific features ensure right strategy from start
\end{itemize}
\end{resultbox}

% ----------------------------------------------------------------------------
\subsection{RouteRAG (Guo et al., 2025)}

\begin{conceptbox}{Paper Details}
\textbf{Title:} RouteRAG: Efficient Retrieval-Augmented Generation from Text and Graph via Reinforcement Learning\\
\textbf{Authors:} Yucan Guo, Miao Su, Saiping Guan, Zihao Sun, Xiaolong Jin, Jiafeng Guo, Xueqi Cheng\\
\textbf{Venue:} arXiv December 2025\\
\textbf{arXiv:} \url{https://arxiv.org/abs/2512.09487}
\end{conceptbox}

\subsubsection{Core Contribution}

RouteRAG uses \textbf{reinforcement learning} to route between:
\begin{itemize}
    \item Text corpus retrieval
    \item Knowledge graph retrieval
    \item Hybrid (reciprocal rank fusion)
\end{itemize}

\subsubsection{Technical Approach}

\begin{itemize}
    \item \textbf{Policy agent:} LLM generates reasoning tokens + retrieval actions
    \item \textbf{Actions:} Retrieve from text, retrieve from KG, terminate with answer
    \item \textbf{Training:} Group-Relative Policy Optimization (GRPO, a PPO variant)
    \item \textbf{Two-stage training:} Optimizes for task outcome AND retrieval efficiency
\end{itemize}

\subsubsection{Key Results}

\begin{itemize}
    \item Outperforms existing RAG baselines on 5 QA benchmarks
    \item Learns when to use expensive graph retrieval vs. cheap text retrieval
    \item Backbone: Qwen2.5-3B and 7B
\end{itemize}

\subsubsection{How We Differ}

\begin{warningbox}{Key Difference}
\textbf{RouteRAG uses RL}; \textbf{we use supervised learning with oracle labels}.

RL challenges:
\begin{itemize}
    \item Reward engineering is tricky
    \item Training instability (PPO requires careful tuning)
    \item Expensive: requires many environment interactions
\end{itemize}

Our supervised approach:
\begin{itemize}
    \item Generate oracle labels once (exhaustive pipeline testing)
    \item Train simple classifier (RandomForest, 5-fold CV)
    \item Stable, reproducible, no reward hacking
\end{itemize}
\end{warningbox}

% ----------------------------------------------------------------------------
\section{Table Understanding Papers}
% ----------------------------------------------------------------------------

\subsection{TableFormer (Nassar et al., CVPR 2022)}

\begin{conceptbox}{Paper Details}
\textbf{Title:} TableFormer: Table Structure Understanding with Transformers\\
\textbf{Authors:} Ahmed Nassar, Nikolaos Livathinos, Maksym Lysak, Peter Staar (IBM Research)\\
\textbf{Venue:} CVPR 2022 (pages 4614-4623)\\
\textbf{arXiv:} \url{https://arxiv.org/abs/2203.01017}
\end{conceptbox}

\subsubsection{Core Contribution}

TableFormer recognizes table structure from images by predicting \textbf{explicit bounding boxes} for table cells alongside structure recognition. This eliminates the need for custom OCR decoders.

\subsubsection{Technical Innovation}

Previous approaches used encoder-dual-decoder (from PubTabNet). TableFormer improves this by:
\begin{enumerate}
    \item Adding an \textbf{object detection decoder} for cell bounding boxes
    \item Enabling direct content extraction from programmatic PDFs
    \item Making the system \textbf{language-agnostic}
\end{enumerate}

\subsubsection{Key Results}

\begin{itemize}
    \item Outperforms all state-of-the-art methods across different datasets
    \item \textbf{97.9\% table structure accuracy} (used in Docling)
    \item Works on complex tables with merged cells, multi-line rows
\end{itemize}

\subsubsection{Why It Matters for Us}

\begin{resultbox}{Critical for Our Breakthrough}
TableFormer (via Docling) is what enabled our +94\% improvement on metrics questions!

\textbf{Before:} PyPDF chunking fragmented tables $\rightarrow$ 27\% metrics accuracy

\textbf{After:} TableFormer preserves table structure $\rightarrow$ 52\% metrics accuracy
\end{resultbox}

% ----------------------------------------------------------------------------
\subsection{TAPAS (Herzig et al., ACL 2020)}

\begin{conceptbox}{Paper Details}
\textbf{Title:} TaPas: Weakly Supervised Table Parsing via Pre-training\\
\textbf{Authors:} Jonathan Herzig, Pawel Krzysztof Nowak, Thomas Müller, Francesco Piccinno, Julian Eisenschlos\\
\textbf{Venue:} ACL 2020 (pages 4320-4333)\\
\textbf{arXiv:} \url{https://arxiv.org/abs/2004.02349}\\
\textbf{HuggingFace:} \url{https://huggingface.co/docs/transformers/en/model_doc/tapas}
\end{conceptbox}

\subsubsection{Core Contribution}

TAPAS answers questions about tables \textbf{without generating logical forms}. Instead, it directly:
\begin{enumerate}
    \item Selects relevant table cells
    \item Optionally applies aggregation (COUNT, SUM, AVERAGE)
\end{enumerate}

\subsubsection{Technical Innovation}

\begin{itemize}
    \item \textbf{Extended BERT architecture} with 7 token types for tabular structure
    \item \textbf{Relative position embeddings} to encode row/column positions
    \item \textbf{Two heads:} cell selection + aggregation prediction
    \item \textbf{Pre-trained} jointly on text and tables crawled from Wikipedia
\end{itemize}

\subsubsection{Key Results}

\begin{itemize}
    \item SQA: 55.1\% $\rightarrow$ 67.2\% accuracy
    \item Competitive with semantic parsing on WikiSQL and WikiTQ
    \item Transfer learning: WikiSQL $\rightarrow$ WikiTQ yields 48.7\% (4.2 points above prior SOTA)
\end{itemize}

\subsubsection{Why It Matters for Us}

TAPAS demonstrates that table understanding can be learned end-to-end. Future work could integrate TAPAS-style table reasoning after retrieval.

% ----------------------------------------------------------------------------
\subsection{TURL (Deng et al., VLDB 2020)}

\begin{conceptbox}{Paper Details}
\textbf{Title:} TURL: Table Understanding through Representation Learning\\
\textbf{Authors:} Xiang Deng, Huan Sun, Alyssa Lees, You Wu, Cong Yu\\
\textbf{Venue:} VLDB 2020 (Vol. 14, No. 3, pages 307-319)\\
\textbf{arXiv:} \url{https://arxiv.org/abs/2006.14806}\\
\textbf{Code:} \url{https://github.com/sunlab-osu/TURL}
\end{conceptbox}

\subsubsection{Core Contribution}

TURL brings the \textbf{pre-training/fine-tuning paradigm} to relational web tables. It learns deep contextualized representations that transfer across 6 table understanding tasks.

\subsubsection{Technical Innovation}

\begin{itemize}
    \item \textbf{Structure-aware Transformer} encoder for row-column structure
    \item \textbf{Masked Entity Recovery (MER)} pre-training objective
    \item Initialized from TinyBERT for efficiency
\end{itemize}

\subsubsection{Benchmark Tasks}

\begin{tabular}{@{}lp{7cm}@{}}
\toprule
\textbf{Task} & \textbf{Description} \\
\midrule
Entity Linking & Link cell mentions to knowledge base entities \\
Column Type Annotation & Predict semantic type of column \\
Relation Extraction & Identify relationships between columns \\
Row Population & Add new rows consistent with table \\
Cell Filling & Complete missing cell values \\
Schema Augmentation & Add new columns to table \\
\bottomrule
\end{tabular}

\subsubsection{Why It Matters for Us}

TURL shows that pre-trained table representations can improve downstream tasks. Future work could use TURL embeddings for better table chunk retrieval.

% ----------------------------------------------------------------------------
\section{Benchmark \& Dataset Papers}
% ----------------------------------------------------------------------------

\subsection{FinanceBench (Islam et al., 2023)}

\begin{conceptbox}{Paper Details}
\textbf{Title:} FinanceBench: A New Benchmark for Financial Question Answering\\
\textbf{Authors:} Pranab Islam, Anand Kannappan, Douwe Kiela, Rebecca Qian, Nino Scherrer, Bertie Vidgen\\
\textbf{Affiliations:} Patronus AI, Contextual AI, Stanford University\\
\textbf{arXiv:} \url{https://arxiv.org/abs/2311.11944}\\
\textbf{Data:} \url{https://github.com/patronus-ai/financebench}
\end{conceptbox}

\subsubsection{Dataset Statistics}

\begin{tabular}{@{}ll@{}}
\toprule
\textbf{Metric} & \textbf{Value} \\
\midrule
Total questions & 10,231 \\
Open-source sample & 150 (used in paper evaluation) \\
Companies covered & 40 publicly traded US companies \\
Documents & 361 public filings (10-K, 10-Q) \\
Time range & 2015-2023 \\
\bottomrule
\end{tabular}

\subsubsection{Question Categories (150 open-source sample)}

\begin{tabular}{@{}lcp{6cm}@{}}
\toprule
\textbf{Category} & \textbf{Count} & \textbf{Description} \\
\midrule
Metrics-generated & 50 & Questions requiring numerical extraction from tables \\
Domain-relevant & 50 & Questions about financial concepts and terminology \\
Novel-generated & 50 & Complex questions requiring reasoning \\
\bottomrule
\end{tabular}

\subsubsection{Key Findings from Original Paper}

\begin{warningbox}{Sobering Results}
\textbf{GPT-4-Turbo with retrieval incorrectly answered or refused 81\% of questions!}

Even state-of-the-art LLMs with external information access are not yet reliable for unsupervised financial analysis.
\end{warningbox}

\subsubsection{Why It Matters for Us}

FinanceBench is our primary evaluation benchmark. Our work specifically targets the \textbf{metrics-generated} category where prior work fails most dramatically.

% ----------------------------------------------------------------------------
\subsection{dsRAG: 96.6\% on FinanceBench}

\begin{conceptbox}{Project Details}
\textbf{Name:} dsRAG (D-Star AI)\\
\textbf{Authors:} Zach and Nick McCormick\\
\textbf{Repository:} \url{https://github.com/D-Star-AI/dsRAG}\\
\textbf{Note:} This is an open-source project, not a peer-reviewed paper
\end{conceptbox}

\subsubsection{Key Result}

dsRAG achieves \textbf{96.6\% accuracy} on FinanceBench using Claude 3.5 Sonnet, compared to 32\% for vanilla RAG.

\subsubsection{Technical Approach}

Three key innovations:
\begin{enumerate}
    \item \textbf{Semantic Segmentation:} Intelligent document sectioning
    \item \textbf{Contextual Auto-generation:} Automatic context enhancement
    \item \textbf{Relevant Segment Extraction (RSE):} Extracting only relevant segments
\end{enumerate}

\subsubsection{Why It Matters for Us}

dsRAG sets the current SOTA on FinanceBench. However:
\begin{itemize}
    \item Uses Claude 3.5 Sonnet (expensive, proprietary)
    \item No learned routing -- uses fixed heuristics
    \item Our goal: achieve competitive results with learned, efficient routing
\end{itemize}

% ----------------------------------------------------------------------------
\section{Document Processing: Docling}
% ----------------------------------------------------------------------------

\begin{conceptbox}{Project Details}
\textbf{Name:} Docling\\
\textbf{Organization:} IBM Research Zurich\\
\textbf{Release:} Open-sourced July 2024\\
\textbf{Website:} \url{https://www.docling.ai/}\\
\textbf{Repository:} \url{https://github.com/docling-project/docling}\\
\textbf{License:} MIT (permissive, open-source)
\end{conceptbox}

\subsubsection{What It Does}

Docling converts PDFs, manuals, and slide decks into structured data (JSON, Markdown) for LLM consumption. Key capabilities:
\begin{itemize}
    \item \textbf{PDF understanding:} Page layout, reading order, table structure
    \item \textbf{Multiple formats:} PDF, DOCX, PPTX, XLSX, HTML, images
    \item \textbf{Code, formulas, image classification} support
\end{itemize}

\subsubsection{Technical Architecture}

Two models power Docling:
\begin{enumerate}
    \item \textbf{Layout model:} Vision model using object detection to dissect page layout
    \item \textbf{TableFormer:} Transforms image-based tables into machine-readable format
\end{enumerate}

Key advantage: \textbf{Avoids OCR} when possible, reducing errors and improving speed by 30$\times$.

\subsubsection{Granite-Docling Model}

IBM released \textbf{Granite-Docling-258M}, an ultra-compact vision-language model:
\begin{itemize}
    \item Only 258M parameters
    \item Apache 2.0 license
    \item Preserves layout, tables, equations, lists
\end{itemize}

\subsubsection{Why It Matters for Us}

\begin{resultbox}{Our Primary Tool}
Docling + TableFormer enabled our breakthrough:
\begin{itemize}
    \item 265 PDFs processed (100\% success)
    \item 89,566 chunks created
    \item 97.9\% table structure accuracy
    \item Metrics questions: 27\% $\rightarrow$ 52\%
\end{itemize}
\end{resultbox}

% ----------------------------------------------------------------------------
\section{RAG Surveys}
% ----------------------------------------------------------------------------

\subsection{Comprehensive RAG Survey (Gao et al., 2024)}

\begin{conceptbox}{Paper Details}
\textbf{Title:} Retrieval-Augmented Generation for Large Language Models: A Survey\\
\textbf{arXiv:} \url{https://arxiv.org/abs/2312.10997}\\
\textbf{Last Updated:} March 2024
\end{conceptbox}

\subsubsection{RAG Evolution Taxonomy}

\begin{tabular}{@{}lp{8cm}@{}}
\toprule
\textbf{Stage} & \textbf{Characteristics} \\
\midrule
Na\"ive RAG & Fixed retrieval, simple concatenation \\
Advanced RAG & Pre/post-retrieval enhancements, better chunking \\
Modular RAG & Interchangeable components, adaptive routing \\
\bottomrule
\end{tabular}

\subsubsection{Why It Matters for Us}

This survey provides the taxonomy for positioning our work. Our contribution falls under \textbf{Modular RAG} with adaptive routing.

% ----------------------------------------------------------------------------
\section{How Our Work Fits the Landscape}

\begin{verbatim}
                    RAG RESEARCH LANDSCAPE

    +-------------------------------------------------------+
    |                    ADAPTIVE RAG                        |
    |                                                        |
    |  +-------------+  +-------------+  +-------------+     |
    |  | Complexity  |  |   Self-     |  |  RL-based   |     |
    |  |   Routing   |  | Correction  |  |   Routing   |     |
    |  | (Adaptive-  |  | (CRAG, Self-|  | (RouteRAG)  |     |
    |  |  RAG 2024)  |  |  RAG 2023)  |  |   2025)     |     |
    |  +-------------+  +-------------+  +-------------+     |
    |                                                        |
    |         +-----------------------------------+          |
    |         |      OUR CONTRIBUTION:            |          |
    |         |  Feature-based Pipeline Routing   |          |
    |         |  + Table-Aware Chunking (Docling) |          |
    |         |  (Domain-specific, supervised,    |          |
    |         |   efficient, interpretable)       |          |
    |         +-----------------------------------+          |
    |                                                        |
    +-------------------------------------------------------+
    |                  FOUNDATIONS                           |
    |  +-------+  +--------+  +----------+  +-----------+   |
    |  |  RAG  |  | MAML   |  | ProtoNets|  |TableFormer|   |
    |  | 2020  |  |  2017  |  |   2017   |  |   2022    |   |
    |  +-------+  +--------+  +----------+  +-----------+   |
    +-------------------------------------------------------+
\end{verbatim}

\section{Recurring Themes from Literature}

\subsection{Pipeline Adaptation vs Model Scaling}

Several papers show that choosing the right strategy for a query can yield gains that pure model scaling might not achieve. This echoes our finding that retrieval engineering can matter more than which LLM you use.

\subsection{Automating the Decision Process}

Different papers use different training signals:
\begin{itemize}
    \item \textbf{Adaptive-RAG}: Auto-labels complexity by model performance
    \item \textbf{RouteRAG}: RL reward combining accuracy and efficiency
    \item \textbf{Self-RAG}: Synthetic data with inserted tokens
    \item \textbf{Ours}: Oracle labels from exhaustive pipeline testing
\end{itemize}

\subsection{Meta-Learning vs Heuristics}

Traditional systems use hard rules (``if question contains number, use calculator''). New approaches replace those with learned policies. Our router learns patterns that hand-crafted rules would miss.

% ============================================================================
% PART 5: DEEP DIVE - CORE METHODS EXPLAINED
% ============================================================================
\newpage
\part{Deep Dive: Core Methods Explained}

This part provides detailed technical explanations of the methods you \textbf{must} understand deeply. Each section includes the mathematical formulation, algorithm pseudocode, and concrete examples.

% ----------------------------------------------------------------------------
\section{MAML: Understanding the Math}
% ----------------------------------------------------------------------------

\subsection{The Problem MAML Solves}

Imagine you have many related tasks (e.g., classifying images of different animal species). Standard training gives you:
\begin{itemize}
    \item A model that works well on training tasks
    \item Poor performance on new tasks (needs lots of data to adapt)
\end{itemize}

MAML asks: \textbf{Can we find a starting point that makes adaptation fast?}

\subsection{Mathematical Formulation}

Let $\theta$ be the model parameters. For a task $\mathcal{T}_i$:

\begin{conceptbox}{Inner Loop: Task Adaptation}
Given support set $D_i^{train}$ from task $\mathcal{T}_i$, compute adapted parameters:
\begin{equation}
\theta'_i = \theta - \alpha \nabla_\theta \mathcal{L}_{\mathcal{T}_i}(f_\theta, D_i^{train})
\end{equation}
where:
\begin{itemize}
    \item $\alpha$ = inner learning rate (typically 0.01)
    \item $\mathcal{L}_{\mathcal{T}_i}$ = loss function for task $i$
    \item $f_\theta$ = model with parameters $\theta$
\end{itemize}
\end{conceptbox}

\begin{conceptbox}{Outer Loop: Meta-Update}
Evaluate adapted parameters on query set $D_i^{test}$, then update $\theta$:
\begin{equation}
\theta \leftarrow \theta - \beta \nabla_\theta \sum_{\mathcal{T}_i \sim p(\mathcal{T})} \mathcal{L}_{\mathcal{T}_i}(f_{\theta'_i}, D_i^{test})
\end{equation}
where:
\begin{itemize}
    \item $\beta$ = outer learning rate (typically 0.001)
    \item Note: We differentiate through $\theta'_i$, which itself depends on $\theta$
\end{itemize}
\end{conceptbox}

\subsection{The Key Insight: Second-Order Gradients}

The outer loop gradient is:
\begin{equation}
\nabla_\theta \mathcal{L}(f_{\theta'_i}) = \nabla_{\theta'_i} \mathcal{L}(f_{\theta'_i}) \cdot \frac{\partial \theta'_i}{\partial \theta}
\end{equation}

Since $\theta'_i = \theta - \alpha \nabla_\theta \mathcal{L}$, we have:
\begin{equation}
\frac{\partial \theta'_i}{\partial \theta} = I - \alpha \nabla^2_\theta \mathcal{L}
\end{equation}

This involves the \textbf{Hessian} $\nabla^2_\theta \mathcal{L}$. Computing this is expensive!

\begin{warningbox}{First-Order Approximation: FOMAML}
In practice, we often ignore the second-order term:
\begin{equation}
\nabla_\theta \mathcal{L}(f_{\theta'_i}) \approx \nabla_{\theta'_i} \mathcal{L}(f_{\theta'_i})
\end{equation}
This is much faster and works nearly as well in practice!
\end{warningbox}

\subsection{Algorithm: Complete Pseudocode}

\begin{lstlisting}[language=Python]
def maml_train(model, task_distribution, num_iterations):
    """
    MAML Training Algorithm

    Args:
        model: Neural network with parameters theta
        task_distribution: p(T) to sample tasks from
        num_iterations: Number of meta-training steps
    """
    theta = model.parameters()  # Initialize randomly

    for iteration in range(num_iterations):
        # Sample batch of tasks
        task_batch = sample_tasks(task_distribution, batch_size=4)

        meta_gradients = []

        for task in task_batch:
            # Get support and query sets
            support_set = task.sample_support(k=5)  # k-shot
            query_set = task.sample_query(n=15)

            # === INNER LOOP ===
            theta_prime = theta.clone()
            for step in range(inner_steps):  # Usually 1-5 steps
                loss = compute_loss(theta_prime, support_set)
                grad = compute_gradient(loss, theta_prime)
                theta_prime = theta_prime - alpha * grad

            # === COMPUTE META-GRADIENT ===
            query_loss = compute_loss(theta_prime, query_set)

            # Option 1: Full MAML (second-order)
            meta_grad = compute_gradient(query_loss, theta)

            # Option 2: FOMAML (first-order, faster)
            # meta_grad = compute_gradient(query_loss, theta_prime)

            meta_gradients.append(meta_grad)

        # === OUTER LOOP: META-UPDATE ===
        avg_meta_grad = average(meta_gradients)
        theta = theta - beta * avg_meta_grad

    return theta
\end{lstlisting}

\subsection{Concrete Example: Router Adaptation}

How MAML applies to our router for cross-domain adaptation:

\begin{lstlisting}[language=Python]
# Task = Domain (Finance, Medical, Legal)
# Support set = 20 labeled (question, best_pipeline) pairs
# Query set = 30 more pairs for evaluation

# INNER LOOP: Adapt to Finance domain
theta_finance = theta.clone()
for step in range(5):
    loss = cross_entropy(router(finance_support, theta_finance), labels)
    theta_finance = theta_finance - 0.01 * gradient(loss)

# Evaluate on Finance query set
finance_query_loss = cross_entropy(router(finance_query, theta_finance), labels)

# Repeat for Medical, Legal...

# OUTER LOOP: Update theta to be good starting point for ALL domains
total_loss = finance_query_loss + medical_query_loss + legal_query_loss
theta = theta - 0.001 * gradient(total_loss)
\end{lstlisting}

% ----------------------------------------------------------------------------
\section{Adaptive-RAG: Complexity-Based Routing (Deep Dive)}
% ----------------------------------------------------------------------------

\subsection{Paper Overview}

\begin{conceptbox}{Adaptive-RAG at a Glance}
\textbf{Core Idea:} Not all questions need the same retrieval strategy. Simple questions don't need retrieval; complex questions need multi-hop reasoning.

\textbf{Solution:} Train a classifier to predict question complexity, then route to the appropriate strategy.

\textbf{Key Innovation:} Automatic label generation from model performance---no human annotation needed.
\end{conceptbox}

\subsection{The Three Retrieval Strategies}

Adaptive-RAG routes questions to one of three strategies:

\begin{verbatim}
+============================================================+
|              ADAPTIVE-RAG: THREE STRATEGIES                 |
+============================================================+

STRATEGY A: NO RETRIEVAL (Simple Questions)
+------------------------------------------+
|  Question: "What is 2 + 2?"              |
|                                          |
|  LLM --> "4"                             |
|                                          |
|  Why: LLM already knows the answer       |
|  Cost: 1 LLM call, 0 retrieval calls     |
+------------------------------------------+

STRATEGY B: SINGLE-STEP RETRIEVAL (Moderate Questions)
+------------------------------------------+
|  Question: "When was the Eiffel Tower    |
|            built?"                       |
|                                          |
|  Retrieve(question) --> [doc1, doc2...]  |
|  LLM(question + docs) --> "1889"         |
|                                          |
|  Why: Single fact lookup                 |
|  Cost: 1 LLM call, 1 retrieval call      |
+------------------------------------------+

STRATEGY C: ITERATIVE RETRIEVAL (Complex Questions)
+------------------------------------------+
|  Question: "How much older is the Eiffel |
|            Tower than the Empire State   |
|            Building?"                    |
|                                          |
|  Step 1: Retrieve("Eiffel Tower built")  |
|          --> 1889                        |
|  Step 2: Retrieve("Empire State built")  |
|          --> 1931                        |
|  Step 3: LLM computes 1931 - 1889 = 42   |
|                                          |
|  Why: Multi-hop reasoning required       |
|  Cost: 1+ LLM calls, 2+ retrieval calls  |
+------------------------------------------+
\end{verbatim}

\subsection{Automatic Label Generation}

The key innovation: use \textbf{model performance} as supervision signal.

\begin{lstlisting}[language=Python]
def generate_complexity_labels(questions, llm, retriever):
    """
    Automatically label questions by complexity.

    The label is determined by which strategy SUCCEEDS:
    - If no-retrieval works --> 'A' (simple)
    - If single-step works but multi-hop fails --> 'B' (moderate)
    - If only multi-hop works --> 'C' (complex)

    Returns:
        labels: List of 'A', 'B', or 'C' per question
    """
    labels = []

    for q in questions:
        gold_answer = get_ground_truth(q)

        # === TRY STRATEGY A: No Retrieval ===
        answer_a = llm.generate(q, context=None)
        correct_a = is_correct(answer_a, gold_answer)

        # === TRY STRATEGY B: Single-Step Retrieval ===
        docs = retriever.retrieve(q, k=5)
        answer_b = llm.generate(q, context=docs)
        correct_b = is_correct(answer_b, gold_answer)

        # === TRY STRATEGY C: Iterative Multi-Hop ===
        answer_c = iterative_rag(q, llm, retriever, max_hops=3)
        correct_c = is_correct(answer_c, gold_answer)

        # === ASSIGN LABEL BASED ON WHAT WORKED ===
        if correct_a:
            labels.append('A')  # Simplest strategy worked
        elif correct_b:
            labels.append('B')  # Needed retrieval, but not multi-hop
        elif correct_c:
            labels.append('C')  # Needed full multi-hop
        else:
            # Nothing worked - default to most powerful strategy
            labels.append('C')

    return labels
\end{lstlisting}

\subsection{The Iterative Retrieval Algorithm}

For complex (Strategy C) questions:

\begin{lstlisting}[language=Python]
def iterative_rag(question, llm, retriever, max_hops=3):
    """
    Iterative retrieval for multi-hop questions.

    At each hop:
    1. Generate a sub-query based on what we know so far
    2. Retrieve relevant documents
    3. Decide if we have enough info to answer
    """
    context = []

    for hop in range(max_hops):
        # Generate sub-query for this hop
        if hop == 0:
            sub_query = question
        else:
            sub_query = llm.generate(
                f"Given the question: {question}\n"
                f"And what we know: {context}\n"
                f"What should we search for next?"
            )

        # Retrieve for this sub-query
        new_docs = retriever.retrieve(sub_query, k=3)
        context.extend(new_docs)

        # Check if we can answer now
        can_answer = llm.generate(
            f"Question: {question}\n"
            f"Context: {context}\n"
            f"Can you answer this question? (yes/no)"
        )

        if can_answer == "yes":
            break

    # Generate final answer with all accumulated context
    answer = llm.generate(
        f"Question: {question}\n"
        f"Context: {context}\n"
        f"Answer:"
    )

    return answer
\end{lstlisting}

\subsection{Classifier Architecture and Training}

\begin{lstlisting}[language=Python]
from transformers import T5ForSequenceClassification

class AdaptiveRAGClassifier:
    """
    T5-based complexity classifier.

    Input: Question text
    Output: Complexity class (A, B, or C)
    """
    def __init__(self):
        # Use T5-base (220M params) - small enough to be fast
        self.model = T5ForSequenceClassification.from_pretrained(
            't5-base',
            num_labels=3  # A=0, B=1, C=2
        )
        self.tokenizer = T5Tokenizer.from_pretrained('t5-base')

    def predict(self, question):
        # Format input with task prefix
        input_text = f"classify query complexity: {question}"
        inputs = self.tokenizer(input_text, return_tensors='pt')

        # Get prediction
        with torch.no_grad():
            logits = self.model(**inputs).logits
            pred_class = torch.argmax(logits, dim=-1).item()

        return ['A', 'B', 'C'][pred_class]

    def train(self, questions, labels, epochs=3):
        """Fine-tune on auto-generated labels."""
        dataset = ClassificationDataset(questions, labels)
        trainer = Trainer(
            model=self.model,
            train_dataset=dataset,
            args=TrainingArguments(
                num_train_epochs=epochs,
                per_device_train_batch_size=16,
                learning_rate=5e-5,
            )
        )
        trainer.train()
\end{lstlisting}

\subsection{Benchmark Results from the Paper}

\begin{tabular}{@{}lccc@{}}
\toprule
\textbf{Method} & \textbf{Single-hop QA} & \textbf{Multi-hop QA} & \textbf{Efficiency} \\
\midrule
No Retrieval & 45.2\% & 12.3\% & Fastest \\
Always Single-Step & 68.4\% & 41.2\% & Medium \\
Always Iterative & 67.1\% & \textbf{52.8\%} & Slowest \\
\midrule
\textbf{Adaptive-RAG} & \textbf{69.2\%} & 51.9\% & \textbf{Adaptive} \\
\bottomrule
\end{tabular}

\vspace{0.5em}
\textit{Key insight: Adaptive-RAG matches or beats the best fixed strategy on each question type while being more efficient overall.}

\subsection{Concrete Examples by Complexity Class}

\begin{tabular}{@{}cp{5cm}p{6cm}@{}}
\toprule
\textbf{Class} & \textbf{Example Question} & \textbf{Why This Classification} \\
\midrule
\textbf{A} (Simple) & ``What is the chemical formula for water?'' & Common knowledge, no retrieval needed \\
\midrule
\textbf{B} (Moderate) & ``What was Tesla's revenue in Q3 2023?'' & Single fact lookup from one document \\
\midrule
\textbf{C} (Complex) & ``Compare Apple's and Microsoft's profit margins for 2023'' & Requires retrieving data for both companies, then computing \\
\bottomrule
\end{tabular}

\subsection{Critical Limitation for Financial QA}

\begin{warningbox}{Why Adaptive-RAG Fails on FinanceBench}
Adaptive-RAG classifies questions by \textbf{reasoning complexity}, not \textbf{retrieval requirements}.

\textbf{Problem Example:}
\begin{verbatim}
Question: "What was Apple's FY2023 operating margin?"

Adaptive-RAG classification: 'B' (single fact lookup)
Adaptive-RAG action: Single-step retrieval

BUT the retrieved chunks may have FRAGMENTED TABLES:
  Chunk 1: "...Operating Income | Net Income..."
  Chunk 2: "...2023 | $114.3B | $97.0B..."

Neither chunk has BOTH the label AND the value!
Result: WRONG ANSWER (even though strategy seems right)
\end{verbatim}
\end{warningbox}

\subsection{Our Key Differentiation}

\begin{resultbox}{Why Our Approach Works Better for Finance}
\textbf{Adaptive-RAG routes by:} Question complexity (simple/moderate/complex)

\textbf{We route by:} Domain-specific features that predict WHAT KIND of retrieval is needed

\begin{tabular}{@{}ll@{}}
\toprule
\textbf{Our Feature} & \textbf{What It Captures} \\
\midrule
\texttt{has\_year} & Likely needs temporal filtering \\
\texttt{has\_metric\_name} & Likely needs TABLE retrieval \\
\texttt{expects\_number} & Answer precision matters \\
\texttt{finance\_density} & Domain-specific vocabulary \\
\bottomrule
\end{tabular}

\vspace{0.5em}
A question can be ``simple'' (single fact) but still need \textbf{table-aware retrieval}. Adaptive-RAG misses this; we don't.
\end{resultbox}

% ----------------------------------------------------------------------------
\section{Self-RAG: Understanding Reflection Tokens (Deep Dive)}
% ----------------------------------------------------------------------------

\subsection{Paper Overview}

\begin{conceptbox}{Self-RAG at a Glance}
\textbf{Core Idea:} Train the LLM to generate special ``reflection tokens'' that guide its own retrieval and generation decisions.

\textbf{Key Innovation:} The model learns to critique itself---deciding when to retrieve, whether retrieved docs are relevant, and whether its answers are supported.

\textbf{Result:} ICLR 2024 Oral (top 1\% of submissions). Outperforms ChatGPT on multiple QA benchmarks.
\end{conceptbox}

\subsection{The Four Reflection Tokens Explained}

Self-RAG extends the LLM vocabulary with four special tokens:

\begin{verbatim}
+================================================================+
|                SELF-RAG REFLECTION TOKENS                       |
+================================================================+

TOKEN 1: [Retrieve] - "Should I look something up?"
+----------------------------------------------------------+
|  Generated: Before producing any output                   |
|  Values: yes | no                                         |
|                                                           |
|  Example:                                                 |
|  Q: "What is 2+2?"      --> [Retrieve]=no (LLM knows)    |
|  Q: "Tesla Q3 revenue?" --> [Retrieve]=yes (need docs)   |
+----------------------------------------------------------+

TOKEN 2: [IsRel] - "Is this document relevant?"
+----------------------------------------------------------+
|  Generated: After each retrieved document                 |
|  Values: relevant | irrelevant                           |
|                                                           |
|  Example:                                                 |
|  Q: "Tesla revenue?"                                      |
|  Doc: "Tesla CEO Elon Musk..." --> [IsRel]=irrelevant    |
|  Doc: "Tesla reported $25B..." --> [IsRel]=relevant      |
+----------------------------------------------------------+

TOKEN 3: [IsSup] - "Is my answer supported by evidence?"
+----------------------------------------------------------+
|  Generated: After producing an answer segment             |
|  Values: fully_supported | partially_supported | no       |
|                                                           |
|  Example:                                                 |
|  Answer: "Tesla's Q3 revenue was $25.2B"                  |
|  Doc says: "Tesla reported revenue of $25.2 billion"     |
|  --> [IsSup]=fully_supported                              |
+----------------------------------------------------------+

TOKEN 4: [IsUse] - "How useful is this response?"
+----------------------------------------------------------+
|  Generated: At the end of the full response               |
|  Values: 1 | 2 | 3 | 4 | 5 (utility score)               |
|                                                           |
|  Example:                                                 |
|  Q: "Explain quantum computing"                           |
|  A: "Quantum computing uses qubits..." (detailed)        |
|  --> [IsUse]=5                                            |
|  A: "It's complicated." (unhelpful)                       |
|  --> [IsUse]=1                                            |
+----------------------------------------------------------+
\end{verbatim}

\subsection{Complete Generation Flow}

\begin{verbatim}
+=================================================================+
|              SELF-RAG GENERATION FLOW                            |
+=================================================================+

Input: "What was Apple's market cap at the end of 2023?"

STEP 1: Decide whether to retrieve
+-----------------------------------+
| Model generates: [Retrieve]=yes   |
| (Financial data needs lookup)     |
+-----------------------------------+
            |
            v
STEP 2: System retrieves documents
+-----------------------------------+
| doc1: "Apple Inc. is a tech..."   |
| doc2: "As of Dec 31, 2023,        |
|        Apple's market cap was     |
|        $2.99 trillion..."         |
| doc3: "Tim Cook became CEO..."    |
+-----------------------------------+
            |
            v
STEP 3: Evaluate each document's relevance
+-----------------------------------+
| doc1: [IsRel]=irrelevant          |
| doc2: [IsRel]=relevant    <--     |
| doc3: [IsRel]=irrelevant          |
+-----------------------------------+
            |
            v
STEP 4: Generate answer using relevant docs
+-----------------------------------+
| "Apple's market cap at the end of |
|  2023 was approximately $2.99     |
|  trillion."                       |
+-----------------------------------+
            |
            v
STEP 5: Check if answer is supported
+-----------------------------------+
| [IsSup]=fully_supported           |
| (Number matches doc2 exactly)     |
+-----------------------------------+
            |
            v
STEP 6: Rate overall utility
+-----------------------------------+
| [IsUse]=5                         |
| (Direct, accurate answer)         |
+-----------------------------------+
            |
            v
FINAL OUTPUT: "Apple's market cap at the end of 2023 was
              approximately $2.99 trillion."
\end{verbatim}

\subsection{Training Pipeline: The Critic Model}

Self-RAG training requires a \textbf{critic model} to generate reflection token labels:

\begin{lstlisting}[language=Python]
def create_self_rag_training_data(corpus, critic_model):
    """
    Create training data with reflection tokens inserted.

    The critic model (GPT-4 or Claude) evaluates each decision
    point and inserts the appropriate reflection token.

    Args:
        corpus: List of (question, passage, answer) tuples
        critic_model: Strong LLM for evaluation (GPT-4)

    Returns:
        training_examples: Text with reflection tokens inserted
    """
    training_examples = []

    for question, passages, gold_answer in corpus:

        # === CRITIC DECISION 1: Should we retrieve? ===
        retrieve_prompt = f"""
        Question: {question}

        Does answering this question require external information
        beyond common knowledge? Answer: yes or no
        """
        retrieve_decision = critic_model.generate(retrieve_prompt)

        if retrieve_decision == "yes":
            # Retrieve documents
            docs = retrieve(question, k=5)

            for doc in docs:
                # === CRITIC DECISION 2: Is doc relevant? ===
                relevance_prompt = f"""
                Question: {question}
                Document: {doc}

                Is this document relevant to answering the question?
                Answer: relevant or irrelevant
                """
                relevance = critic_model.generate(relevance_prompt)

                if relevance == "relevant":
                    # Generate answer conditioned on this doc
                    answer = generate_answer(question, doc)

                    # === CRITIC DECISION 3: Is answer supported? ===
                    support_prompt = f"""
                    Document: {doc}
                    Generated answer: {answer}

                    Is the answer supported by the document?
                    Answer: fully_supported, partially_supported, or no_support
                    """
                    support = critic_model.generate(support_prompt)

                    # === CRITIC DECISION 4: How useful? ===
                    utility_prompt = f"""
                    Question: {question}
                    Answer: {answer}

                    Rate the utility of this answer from 1-5.
                    """
                    utility = critic_model.generate(utility_prompt)

                    # === CONSTRUCT TRAINING EXAMPLE ===
                    training_example = (
                        f"Question: {question}\n"
                        f"[Retrieve]yes\n"
                        f"Document: {doc}\n"
                        f"[IsRel]relevant\n"
                        f"Answer: {answer}\n"
                        f"[IsSup]{support}\n"
                        f"[IsUse]{utility}"
                    )
                    training_examples.append(training_example)

        else:
            # No retrieval needed
            answer = generate_answer(question, context=None)
            training_example = (
                f"Question: {question}\n"
                f"[Retrieve]no\n"
                f"Answer: {answer}\n"
                f"[IsUse]{utility}"
            )
            training_examples.append(training_example)

    return training_examples
\end{lstlisting}

\subsection{Inference: Critique-Guided Beam Search}

At test time, Self-RAG uses reflection tokens to \textbf{score and select} generation paths:

\begin{lstlisting}[language=Python]
def self_rag_inference(question, model, retriever, num_beams=4):
    """
    Self-RAG inference with critique-guided beam search.

    The model generates reflection tokens, which are used to
    score different generation paths. Best path wins.
    """

    # Step 1: Should we retrieve?
    retrieve_logits = model.get_next_token_logits("[Retrieve]")
    p_retrieve = softmax(retrieve_logits)['yes']

    if p_retrieve > 0.5:
        # Retrieve documents
        docs = retriever.retrieve(question, k=5)

        # Score each document for relevance
        candidates = []
        for doc in docs:
            # Get relevance score
            rel_logits = model.get_next_token_logits(
                f"Question: {question}\nDoc: {doc}\n[IsRel]"
            )
            rel_score = softmax(rel_logits)['relevant']

            if rel_score > 0.5:  # Keep relevant docs
                # Generate answer with this doc
                answer = model.generate(
                    f"Question: {question}\nDoc: {doc}\nAnswer:"
                )

                # Get support score
                sup_logits = model.get_next_token_logits(
                    f"Doc: {doc}\nAnswer: {answer}\n[IsSup]"
                )
                sup_score = softmax(sup_logits)['fully_supported']

                # Get utility score
                use_logits = model.get_next_token_logits(
                    f"Q: {question}\nA: {answer}\n[IsUse]"
                )
                use_score = expected_value(softmax(use_logits))  # 1-5

                # Combined score
                total_score = rel_score + sup_score + use_score / 5
                candidates.append((answer, total_score, doc))

        # Return best candidate
        candidates.sort(key=lambda x: x[1], reverse=True)
        return candidates[0][0] if candidates else "Unable to answer."

    else:
        # Generate without retrieval
        return model.generate(f"Question: {question}\nAnswer:")
\end{lstlisting}

\subsection{Key Results from the Paper}

\begin{tabular}{@{}lcccc@{}}
\toprule
\textbf{Model} & \textbf{PopQA} & \textbf{TriviaQA} & \textbf{Biography} & \textbf{ASQA} \\
\midrule
Llama2-7B & 14.7 & 55.8 & -- & -- \\
Llama2-7B + RAG & 38.2 & 62.3 & -- & -- \\
ChatGPT & 29.3 & 63.4 & 71.2 & 35.3 \\
ChatGPT + RAG & 50.8 & 67.2 & 74.1 & 40.1 \\
\midrule
\textbf{Self-RAG 7B} & \textbf{54.9} & 66.4 & \textbf{81.2} & \textbf{42.8} \\
\textbf{Self-RAG 13B} & 53.8 & \textbf{69.3} & 79.4 & 41.6 \\
\bottomrule
\end{tabular}

\vspace{0.5em}
\textit{Self-RAG 7B outperforms ChatGPT+RAG on 3/4 benchmarks while being much smaller!}

\subsection{Training Costs and Requirements}

\begin{tabular}{@{}ll@{}}
\toprule
\textbf{Requirement} & \textbf{Details} \\
\midrule
Base Model & Llama 2 (7B or 13B) \\
Critic Model & GPT-4 for label generation \\
Training Data & 150K examples with reflection tokens \\
Fine-tuning & Full parameter fine-tuning (not LoRA) \\
GPU Hours & $\sim$100 A100 hours for 7B model \\
Vocab Extension & +4 special tokens \\
\bottomrule
\end{tabular}

\subsection{Critical Limitation for Our Use Case}

\begin{warningbox}{Why Self-RAG Doesn't Fit Our Needs}
Self-RAG requires \textbf{modifying the LLM itself}. This creates several problems:

\textbf{1. Model Lock-in:}
\begin{itemize}
    \item Fine-tuned on Llama 2 $\rightarrow$ Can't use GPT-4, Claude, or newer Llamas
    \item Each new base model requires full retraining
\end{itemize}

\textbf{2. High Cost:}
\begin{itemize}
    \item Need GPT-4 to generate 150K training labels
    \item Need 100+ A100 GPU hours for fine-tuning
    \item Need to maintain custom model weights
\end{itemize}

\textbf{3. Reflection $\neq$ Domain Expertise:}
\begin{itemize}
    \item Self-RAG learns WHEN to retrieve, not WHAT to retrieve
    \item It doesn't know that financial questions need table-aware retrieval
    \item A revenue question still gets generic semantic search
\end{itemize}
\end{warningbox}

\subsection{Our Key Differentiation}

\begin{resultbox}{Why Our Approach is Better for Production}

\begin{tabular}{@{}p{3.5cm}p{4.5cm}p{4.5cm}@{}}
\toprule
\textbf{Aspect} & \textbf{Self-RAG} & \textbf{Our Approach} \\
\midrule
LLM modification & Required (fine-tuning) & \textbf{None} \\
Model flexibility & Locked to one base model & \textbf{Any LLM works} \\
Training cost & High (GPT-4 + fine-tuning) & \textbf{Low (sklearn)} \\
Update routing & Retrain entire model & \textbf{Retrain small classifier} \\
Domain awareness & Generic & \textbf{21 domain features} \\
Table handling & No special support & \textbf{Docling + TableFormer} \\
\bottomrule
\end{tabular}

\vspace{0.5em}
\textbf{Bottom line:} Self-RAG optimizes WHEN to retrieve. We optimize WHAT KIND of retrieval to use---which matters more for domain-specific tasks like financial QA.
\end{resultbox}

% ----------------------------------------------------------------------------
\section{TableFormer: How It Preserves Table Structure}
% ----------------------------------------------------------------------------

\subsection{The Problem with Tables in PDFs}

When a PDF parser encounters a table:
\begin{verbatim}
+--------+---------+--------+
| Year   | Revenue | Growth |
+--------+---------+--------+
| 2021   | $365B   | 33%    |
| 2022   | $394B   | 8%     |
| 2023   | $383B   | -3%    |
+--------+---------+--------+

Standard OCR output (WRONG):
"Year Revenue Growth 2021 $365B 33% 2022 $394B 8% 2023 $383B -3%"

--> All structure is LOST!
--> "What was 2023 growth?" cannot be answered correctly
\end{verbatim}

\subsection{TableFormer Architecture}

TableFormer uses a \textbf{transformer encoder-decoder} with two specialized heads:

\begin{verbatim}
                TableFormer Architecture

    PDF Page Image (RGB)
           |
           v
    +----------------+
    | Vision Encoder |  (ResNet or ViT backbone)
    | (Extract CNN   |
    |  features)     |
    +----------------+
           |
           v
    +----------------+
    | Transformer    |
    | Encoder        |  (Self-attention over image patches)
    +----------------+
           |
     +-----+-----+
     |           |
     v           v
+----------+ +----------+
| Structure| | Bounding |
| Decoder  | | Box Head |
|          | | (DETR-   |
|(Predicts | | style)   |
|table HTML| |          |
|structure)| |(Predicts |
+----------+ |cell bbox)|
     |       +----------+
     v            |
"<table>          v
 <tr>         (x1,y1,x2,y2)
  <td>        for each cell
  ..."
\end{verbatim}

\subsection{Two-Stage Output}

\textbf{Stage 1: Structure Prediction}
\begin{lstlisting}[language=html]
<table>
  <thead>
    <tr>
      <th>Year</th>
      <th>Revenue</th>
      <th>Growth</th>
    </tr>
  </thead>
  <tbody>
    <tr>
      <td></td>  <!-- Content filled in Stage 2 -->
      <td></td>
      <td></td>
    </tr>
    ...
  </tbody>
</table>
\end{lstlisting}

\textbf{Stage 2: Cell Content Extraction}

For each cell, the bounding box head provides coordinates. Then:
\begin{enumerate}
    \item If PDF is programmatic: Extract text from PDF at those coordinates
    \item If PDF is scanned: Run OCR only on that bounding box
\end{enumerate}

\subsection{Why Bounding Boxes Matter}

\begin{conceptbox}{Key Innovation}
Previous methods used a \textbf{character-level decoder} to generate cell content.

Problem: OCR errors compound!

TableFormer's approach:
\begin{itemize}
    \item Predict structure and bounding boxes separately
    \item Extract content directly from source (no OCR if possible)
    \item Errors don't propagate
\end{itemize}
\end{conceptbox}

\subsection{Training Objective}

\begin{equation}
\mathcal{L}_{total} = \mathcal{L}_{structure} + \lambda \mathcal{L}_{bbox}
\end{equation}

Where:
\begin{itemize}
    \item $\mathcal{L}_{structure}$ = Cross-entropy loss for HTML token prediction
    \item $\mathcal{L}_{bbox}$ = L1 loss + GIoU loss for bounding boxes (DETR-style)
    \item $\lambda$ = Balancing weight (typically 1.0)
\end{itemize}

\subsection{Our Usage via Docling}

\begin{lstlisting}[language=Python]
from docling import process_pdf

def ingest_with_tableformer(pdf_path):
    """
    Process PDF preserving table structure.
    """
    result = process_pdf(
        pdf_path,
        table_mode='accurate',  # Use TableFormer
        chunk_strategy='semantic'
    )

    chunks = []
    for element in result.elements:
        if element.type == 'table':
            # Table preserved as markdown!
            table_md = element.to_markdown()
            # Example output:
            # | Year | Revenue | Growth |
            # |------|---------|--------|
            # | 2021 | $365B   | 33%    |
            # | 2022 | $394B   | 8%     |

            # Store as atomic chunk (never split)
            chunks.append({
                'content': table_md,
                'type': 'table',
                'metadata': element.metadata
            })
        else:
            chunks.append({
                'content': element.text,
                'type': 'prose'
            })

    return chunks
\end{lstlisting}

% ----------------------------------------------------------------------------
\section{Prototypical Networks: Simpler Alternative to MAML}
% ----------------------------------------------------------------------------

\subsection{Core Intuition}

Instead of learning ``how to adapt'' (MAML), learn ``how to compare''.

\begin{verbatim}
                  Embedding Space

         * * *     (Class A examples)
          \|/
           *  <-- Prototype A (mean)

                   ? <-- New query

           *  <-- Prototype B (mean)
          /|\
         * * *     (Class B examples)

Query is closer to Prototype A --> Classify as A
\end{verbatim}

\subsection{Mathematical Formulation}

\textbf{Step 1: Compute Class Prototypes}

Given support set $S$ with classes $\{1, ..., K\}$:
\begin{equation}
c_k = \frac{1}{|S_k|} \sum_{(x_i, y_i) \in S_k} f_\phi(x_i)
\end{equation}

where $f_\phi$ is an embedding network (e.g., CNN or Transformer).

\textbf{Step 2: Classify by Distance}

For query point $x$:
\begin{equation}
p(y=k | x) = \frac{\exp(-d(f_\phi(x), c_k))}{\sum_{k'} \exp(-d(f_\phi(x), c_{k'}))}
\end{equation}

\begin{warningbox}{Distance Function Matters!}
The paper finds \textbf{Euclidean distance} significantly outperforms cosine similarity:
\begin{equation}
d(z, c_k) = ||z - c_k||^2 \quad \text{(Euclidean)}
\end{equation}

This is counterintuitive -- most NLP uses cosine. But for few-shot, Euclidean in the learned space works better.
\end{warningbox}

\subsection{Training: Episodic Learning}

\begin{lstlisting}[language=Python]
def train_protonet(embedding_net, episodes, num_iterations):
    """
    Train ProtoNet with episodic learning.

    Each episode simulates a few-shot task.
    """
    optimizer = Adam(embedding_net.parameters(), lr=1e-3)

    for iteration in range(num_iterations):
        # Sample random episode
        # N classes, K examples per class (support)
        # Q queries per class
        support, query, labels = sample_episode(
            episodes, n_way=5, k_shot=5, n_query=15
        )

        # Embed support examples
        support_embeddings = embedding_net(support)  # [N*K, D]

        # Reshape and compute prototypes
        support_embeddings = support_embeddings.view(N, K, D)
        prototypes = support_embeddings.mean(dim=1)  # [N, D]

        # Embed queries
        query_embeddings = embedding_net(query)  # [N*Q, D]

        # Compute distances to all prototypes
        # dists[i,j] = distance from query i to prototype j
        dists = euclidean_distance(query_embeddings, prototypes)

        # Negative distance -> logits (closer = higher prob)
        logits = -dists

        # Cross-entropy loss
        loss = F.cross_entropy(logits, labels)

        # Backprop
        optimizer.zero_grad()
        loss.backward()
        optimizer.step()

    return embedding_net
\end{lstlisting}

\subsection{ProtoNet for Our Router}

\begin{lstlisting}[language=Python]
class ProtoNetRouter:
    """
    Route questions to pipelines using prototypical approach.
    """
    def __init__(self, embedding_model):
        self.embedder = embedding_model
        self.prototypes = {}  # pipeline -> prototype vector

    def fit(self, oracle_labels):
        """
        Compute prototype for each pipeline from oracle data.
        """
        pipeline_questions = defaultdict(list)

        for q_id, data in oracle_labels.items():
            best_pipeline = data['best_pipeline']
            question_text = data['question']
            pipeline_questions[best_pipeline].append(question_text)

        for pipeline, questions in pipeline_questions.items():
            # Embed all questions for this pipeline
            embeddings = self.embedder.encode(questions)
            # Prototype = mean embedding
            self.prototypes[pipeline] = embeddings.mean(axis=0)

    def route(self, question):
        """
        Route by finding nearest prototype.
        """
        q_embed = self.embedder.encode(question)

        min_dist = float('inf')
        best_pipeline = None

        for pipeline, prototype in self.prototypes.items():
            dist = np.linalg.norm(q_embed - prototype)  # Euclidean
            if dist < min_dist:
                min_dist = dist
                best_pipeline = pipeline

        return best_pipeline
\end{lstlisting}

\subsection{ProtoNet vs MAML: When to Use Which}

\begin{tabular}{@{}p{3cm}p{5cm}p{5cm}@{}}
\toprule
\textbf{Aspect} & \textbf{ProtoNet} & \textbf{MAML} \\
\midrule
Complexity & Simple (no inner loop) & Complex (nested optimization) \\
Training & Fast (just embeddings) & Slow (second-order gradients) \\
Adaptation & Fixed (just recompute prototypes) & Flexible (gradient steps) \\
When to use & Static domains & Evolving domains \\
Our case & Good for initial router & Better for cross-domain transfer \\
\bottomrule
\end{tabular}

% ----------------------------------------------------------------------------
\section{Putting It All Together: Our Method}
% ----------------------------------------------------------------------------

\subsection{Complete Pipeline Architecture}

\begin{verbatim}
+=====================================================================+
|                    OUR TABLE-AWARE ROUTED RAG                       |
+=====================================================================+

                         User Question
                              |
                              v
    +---------------------------------------------------+
    |  STAGE 1: FEATURE EXTRACTION (21 features)        |
    |                                                    |
    |  - Temporal: has_year, has_quarter, fiscal_ind   |
    |  - Entity: company_indicator, metric_name         |
    |  - Structure: expects_number, is_what/how/why     |
    |  - Domain: finance_density, medical_density       |
    +---------------------------------------------------+
                              |
                              v
    +---------------------------------------------------+
    |  STAGE 2: LEARNED ROUTER (RandomForest/MLP)       |
    |                                                    |
    |  Input: 21 features                                |
    |  Output: Best (pipeline, top_k) config            |
    |                                                    |
    |  Trained on oracle labels from exhaustive search  |
    +---------------------------------------------------+
                              |
                              v
    +---------------------------------------------------+
    |  STAGE 3: TABLE-AWARE RETRIEVAL                   |
    |                                                    |
    |  ChromaDB (built with Docling + TableFormer)      |
    |  - 89,566 chunks                                   |
    |  - Tables preserved as atomic units               |
    |  - 97.9% table structure accuracy                 |
    +---------------------------------------------------+
                              |
                              v
    +---------------------------------------------------+
    |  STAGE 4: LLM GENERATION                          |
    |                                                    |
    |  Any off-the-shelf LLM (GPT-4o-mini, Llama 70B)  |
    |  Context: Retrieved chunks (including tables)     |
    +---------------------------------------------------+
                              |
                              v
                         Final Answer
\end{verbatim}

\subsection{Training the Router: Step by Step}

\begin{lstlisting}[language=Python]
# === Step 1: Generate Oracle Labels ===
oracle_labels = {}
for question in financebench_questions:
    best_score = 0
    best_config = None

    for pipeline in ['semantic', 'hybrid_filter_rerank']:
        for top_k in [5, 10, 20]:
            answer = run_pipeline(question, pipeline, top_k)
            score = evaluate(answer, ground_truth)

            if score > best_score:
                best_score = score
                best_config = (pipeline, top_k)

    oracle_labels[question.id] = best_config

# === Step 2: Extract Features ===
X = []  # Feature matrix
y = []  # Labels (pipeline index)

for q_id, (pipeline, top_k) in oracle_labels.items():
    features = extract_21_features(questions[q_id])
    X.append(features)
    y.append(pipeline_to_idx[f"{pipeline}_k{top_k}"])

X = np.array(X)  # [150, 21]
y = np.array(y)  # [150,]

# === Step 3: Train Classifier ===
from sklearn.ensemble import RandomForestClassifier
from sklearn.model_selection import cross_val_score

router = RandomForestClassifier(n_estimators=100, max_depth=10)
scores = cross_val_score(router, X, y, cv=5)
print(f"5-fold CV accuracy: {scores.mean():.2%}")

router.fit(X, y)
joblib.dump(router, 'models/pipeline_router.pkl')

# === Step 4: Inference ===
def answer_question(question):
    # Extract features (< 1ms)
    features = extract_21_features(question)

    # Predict best config (< 1ms)
    config_idx = router.predict([features])[0]
    pipeline, top_k = idx_to_config[config_idx]

    # Run only the predicted pipeline (1 API call)
    return run_pipeline(question, pipeline, top_k)
\end{lstlisting}

\subsection{Why This Works Better Than Alternatives}

\begin{resultbox}{Our Advantages}
\textbf{vs Adaptive-RAG:}
\begin{itemize}
    \item We route on domain features, not just complexity
    \item Finance questions need table awareness, not more retrieval iterations
\end{itemize}

\textbf{vs Self-RAG:}
\begin{itemize}
    \item No LLM modification required
    \item Works with any off-the-shelf model
    \item Much cheaper to update routing logic
\end{itemize}

\textbf{vs RouteRAG:}
\begin{itemize}
    \item Supervised learning is more stable than RL
    \item Oracle labels provide clear training signal
    \item No reward engineering needed
\end{itemize}

\textbf{Plus: Table-Aware Chunking (Docling)}
\begin{itemize}
    \item 97.9\% table structure accuracy
    \item Tables never split across chunks
    \item +94\% improvement on metrics questions
\end{itemize}
\end{resultbox}

% ============================================================================
% PART 6: TESTING & ITERATION WORKFLOW
% ============================================================================
\newpage
\part{Testing \& Iteration Workflow}

\section{The Iteration Cycle}

\begin{verbatim}
+---------------------------------------------------------------+
|                    ITERATION CYCLE                             |
|                                                                |
|    +--------+    +--------+    +---------+    +--------+       |
|    | Change |--->|  Test  |--->| Analyze |--->| Decide |       |
|    |Something    |(150 Q's)    | Results |    |Next Fix|       |
|    +--------+    +--------+    +---------+    +--------+       |
|         ^                                          |           |
|         +------------------------------------------+           |
+---------------------------------------------------------------+
\end{verbatim}

\section{Step 1: Make a Change}

Each iteration tests ONE hypothesis:

\begin{tabular}{@{}clp{6cm}@{}}
\toprule
\textbf{Iteration} & \textbf{Hypothesis} & \textbf{What We Changed} \\
\midrule
1 & ``More context helps'' & top-k: 10 $\rightarrow$ 20 $\rightarrow$ 30 \\
2 & ``Reranking helps'' & Pipeline: semantic $\rightarrow$ hybrid\_filter\_rerank \\
3 & ``Better model helps'' & Model: Llama 3.1 $\rightarrow$ Qwen $\rightarrow$ GPT-4o-mini \\
4 & ``Table chunking is key'' & Ingestion: PyPDF $\rightarrow$ Docling+TableFormer \\
\bottomrule
\end{tabular}

\section{Step 2: Run Evaluation}

\begin{lstlisting}[language=bash]
python src/bulk_testing.py \
    --model gpt-4o-mini \
    --pipeline semantic \
    --top-k 10 \
    --embedding openai-large \
    --chroma-path chroma_docling
\end{lstlisting}

What happens for each of 150 questions:
\begin{enumerate}
    \item \textbf{RETRIEVE}: Query ChromaDB $\rightarrow$ Get top-k documents
    \item \textbf{GENERATE}: Send context + question to LLM $\rightarrow$ Get answer
    \item \textbf{EVALUATE}: Compare answer to gold standard
\end{enumerate}

\section{Step 3: Analyze Results}

\begin{lstlisting}[language=bash]
python scripts/analyze_results.py --results bulk_runs/latest.csv
python scripts/error_analysis.py --results bulk_runs/latest.csv
\end{lstlisting}

\section{Step 4: Decide Next Fix}

\begin{verbatim}
IF metrics-generated is low:
    --> Problem is likely TABLE HANDLING
    --> Try: better chunking, larger chunks, table-aware parsing

IF all categories are equally low:
    --> Problem is likely MODEL or RETRIEVAL
    --> Try: different model, more top-k, reranking

IF one company/doc consistently fails:
    --> Problem is likely INGESTION for that doc
    --> Try: re-ingest, check OCR quality
\end{verbatim}

\section{Our Iteration History}

\subsection{Iterations 1-10: Finding the Ceiling}

We tried 8 variations (top-k, pipeline, model) and gained only $\sim$1\%. This PROVED that retrieval tuning and model choice weren't the problem.

\begin{warningbox}{Key Learning}
Without these ``failed'' iterations, we might have kept tuning the wrong thing. The negative results were essential to identifying the real bottleneck.
\end{warningbox}

\subsection{Iteration 11: The Breakthrough}

Changed from PyPDF chunking to Docling with TableFormer:
\begin{itemize}
    \item Overall: 48.5\% $\rightarrow$ 60.4\%
    \item Metrics: 26.7\% $\rightarrow$ 51.9\%
\end{itemize}

% ============================================================================
% PART 6: FUTURE WORK & EXTENSIONS
% ============================================================================
\newpage
\part{Future Work \& Extensions}

\section{Improving the Router}

\subsection{Level 1: Better Features (Easy Win)}

Add question embeddings as features:
\begin{lstlisting}[language=Python]
# Current: 21 hand-crafted features
features = extract_features(question)  # [has_year, has_company, ...]

# Better: 768-dim dense embedding
embedding = embed_model.encode(question)
features = np.concatenate([hand_crafted, embedding])
\end{lstlisting}

\subsection{Level 2: Neural Network Classifiers}

Replace RandomForest with MLP:
\begin{lstlisting}[language=Python]
class RouterMLP(nn.Module):
    def __init__(self, embed_dim=1024, num_classes=6):
        self.net = nn.Sequential(
            nn.Linear(embed_dim, 256),
            nn.ReLU(),
            nn.Dropout(0.3),
            nn.Linear(256, 64),
            nn.ReLU(),
            nn.Linear(64, num_classes)
        )
\end{lstlisting}

\subsection{Level 3: Fine-tuned Small Transformer}

Use MiniLM (22M params) fine-tuned for routing:
\begin{lstlisting}[language=Python]
from transformers import AutoModelForSequenceClassification

model = AutoModelForSequenceClassification.from_pretrained(
    "microsoft/MiniLM-L6-v2",
    num_labels=6  # Pipeline classes
)
\end{lstlisting}

\subsection{Level 4: MAML for Cross-Domain}

Train router to adapt quickly to new domains with few examples. This is the strongest future work direction for a full paper.

\section{Other Extensions}

\subsection{Numeric Verification}

Add step to verify numbers in generated answer exist in retrieved context:
\begin{lstlisting}[language=Python]
def verify_answer(answer, retrieved_docs):
    answer_numbers = extract_numbers(answer)
    context_numbers = extract_numbers(docs)

    for num in answer_numbers:
        if num not in context_numbers:
            flag_hallucination(num)
\end{lstlisting}

\subsection{Multi-Domain Datasets}

Expand beyond finance:
\begin{itemize}
    \item \textbf{Medical}: PubMedQA
    \item \textbf{Legal}: CUAD (contract understanding)
    \item \textbf{Scientific}: Papers with tables
\end{itemize}

% ============================================================================
% PART 7: TECHNICAL DETAILS
% ============================================================================
\newpage
\part{Technical Details}

\section{Cluster Setup (Together AI)}

\subsection{Environment Variables}

\begin{lstlisting}[language=bash]
# Cache directories (cluster /home is read-only!)
export HF_HOME=/data/junjiexiong/.cache/huggingface
export TRANSFORMERS_CACHE=/data/junjiexiong/.cache/huggingface
export SENTENCE_TRANSFORMERS_HOME=/data/junjiexiong/.cache/huggingface
export UV_CACHE_DIR=/data/junjiexiong/.cache/uv
export PIP_CACHE_DIR=/data/junjiexiong/.cache/pip

# API keys
export OPENAI_API_KEY="your-key"
export TOGETHER_API_KEY="your-key"
\end{lstlisting}

\subsection{SLURM Commands}

\begin{tabular}{@{}ll@{}}
\toprule
\textbf{Task} & \textbf{Command} \\
\midrule
Submit job & \texttt{sbatch scripts/eval\_job.sh} \\
Check status & \texttt{squeue -u junjiexiong} \\
Cancel job & \texttt{scancel <job\_id>} \\
Interactive GPU & \texttt{srun --gres=gpu:1 --mem=32G --pty bash} \\
View logs & \texttt{tail -f logs/eval\_*.out} \\
\bottomrule
\end{tabular}

\section{Fixes Applied During Development}

\begin{longtable}{@{}p{2cm}p{4cm}p{6cm}@{}}
\toprule
\textbf{Date} & \textbf{Fix} & \textbf{Details} \\
\midrule
\endhead
Session 1 & Created evaluation module & Semantic similarity, LLM-as-judge \\
Session 1 & Created dataset adapters & FinanceBench, PubMedQA loaders \\
Session 1 & ChromaDB cleanup & Removed 8,321 duplicate/empty chunks \\
Session 2 & Model routing fix & ``Turbo'' models $\rightarrow$ Together API \\
Session 3 & Embedding CLI arg & Fixed 3072d vs 1024d mismatch \\
Session 3 & rank\_bm25 missing & Added to requirements.txt \\
Session 3 & pyarrow version & Upgraded to fix mismatch \\
\bottomrule
\end{longtable}

\section{Project Structure}

\begin{lstlisting}
rag/
|-- src/
|   |-- config.py              # All configuration
|   |-- bulk_testing.py        # Main evaluation script
|   |-- ingest.py              # PDF ingestion
|   |-- providers/             # LLM provider adapters
|   |-- retrieval_tools/       # Retrieval pipelines
|   +-- meta_learning/
|       |-- features.py        # 21 feature extractor
|       +-- router.py          # Router class
|
|-- scripts/
|   |-- generate_oracle_labels.py
|   |-- train_router.py
|   |-- error_analysis.py
|   +-- analyze_results.py
|
|-- dataset_adapters/
|   |-- financebench.py
|   +-- pubmedqa.py
|
|-- evaluation/
|   |-- metrics.py
|   +-- llm_judge.py
|
|-- data/
|   +-- question_sets/
|       +-- financebench_open_source.jsonl
|
|-- chroma_docling/            # Vector database
+-- bulk_runs/                 # Result CSVs
\end{lstlisting}

% ============================================================================
% END
% ============================================================================

\vspace{2em}
\hrule
\vspace{1em}

\begin{center}
\textit{End of Research Notes}\\
\textit{Last compiled: \today}
\end{center}

\end{document}
